\documentclass[../../main.tex]{subfiles}
\graphicspath{{\subfix{../images/}}} % 指定图片目录,后续可以直接使用图片文件名。
\begin{document}
\section{Homework 3}
\subsection{Schwinger boson representation}
\textbf{A two-dimensional quantum harmonic oscillator contains two decoupled free bosons, whose annihilation operators can be represented as $a$ and $b$ respectively. $\begin{aligned}
  a = \frac{1}{\sqrt{2}}(x + ip_{x})
\end{aligned}$, $\begin{aligned}
  b = \frac{1}{\sqrt{2}}(y + ip_{y})
\end{aligned}$. They satisfy the commutation relations $[a, a^{\dagger}] = [b, b^{\dagger}] = 1$ and $[a, b] = [a, b^{\dagger}] = 0$. This system has U(2) symmetry, which includes an SU(2) subgroup. Let's explore how to construct the SU(2) representation using bosonic operators. Define $\begin{aligned}
  S^{x}=\frac{1}{2}(a^{\dagger}b + b^{\dagger}a)
\end{aligned}$, $\begin{aligned}
  S^{z} = \frac{1}{2}(a^{\dagger}a - b^{\dagger}b)
\end{aligned}$.}
\begin{enumerate}
  \item \textbf{Express $S^{y}$ in terms of $a$ and $b$. [Hint: Make $\vec{S}\times\vec{S} = i\vec{S}$]}

{\color{gray}{To satisfy the commutation relation $\vec{S}\times\vec{S} = i\vec{S}$, we have
\begin{align*}
  [S^{x},S^{y}] = iS^{z},\quad [S^{y},S^{z}] = iS^{x},\quad [S^{z},S^{x}] = iS^{y}
\end{align*}
So we have
\begin{align*}
  S^{y} &= \frac{1}{i}[S^{z},S^{x}] = \frac{1}{i}\left[\frac{1}{2}\left(a^{\dagger}a - b^{\dagger}b\right), \frac{1}{2}\left(a^{\dagger}b + b^{\dagger}a\right)\right]\\
  &= \frac{1}{4i}[a^{\dagger}a - b^{\dagger}b, a^{\dagger}b + b^{\dagger}a]
\end{align*}
We have commutation formula that
\begin{align*}
  [\hat{A},\hat{B}+\hat{C}] &= [\hat{A},\hat{B}] + [\hat{A},\hat{C}]\\
  [\hat{A}+\hat{B},\hat{C}] &= [\hat{A},\hat{C}] + [\hat{B},\hat{C}]\\
  [\hat{A},\hat{B}\hat{C}] &= \hat{B}[\hat{A},\hat{C}] + [\hat{A},\hat{B}]\hat{C}\\
  [\hat{A}\hat{B},\hat{C}] &= \hat{A}[\hat{B},\hat{C}] + [\hat{A},\hat{C}]\hat{B}\\
  \Rightarrow [\hat{A}\hat{B},\hat{C}\hat{D}] &= \hat{A}\hat{C}[\hat{B},\hat{D}] + \hat{A}[\hat{B},\hat{C}]\hat{D} + \hat{C}[\hat{A},\hat{D}]\hat{B} + [\hat{A},\hat{C}]\hat{D}\hat{B}
\end{align*}
So we have
\begin{align*}
  S^{y} &= \frac{1}{4i}\left[a^{\dagger}a, a^{\dagger}b\right] + \frac{1}{4i}\left[a^{\dagger}a, b^{\dagger}a\right] - \frac{1}{4i}\left[b^{\dagger}b, a^{\dagger}b\right] - \frac{1}{4i}\left[b^{\dagger}b, b^{\dagger}a\right]\\
  \left[a^{\dagger}a, a^{\dagger}b\right] &= \cancel{a^{\dagger}a^{\dagger}[a,b]} + a^{\dagger}[a,a^{\dagger}]b + \cancel{a^{\dagger}[a^{\dagger},b]a} + \cancel{[a^{\dagger},a^{\dagger}]ba} = a^{\dagger}b\\
  \left[a^{\dagger}a, b^{\dagger}a\right] &= \cancel{a^{\dagger}b^{\dagger}[a,a]} + \cancel{a^{\dagger}[a,b^{\dagger}]a} + b^{\dagger}[a^{\dagger},a]a + \cancel{[a^{\dagger},b^{\dagger}]aa} = -b^{\dagger}a\\
  \left[b^{\dagger}b, a^{\dagger}b\right] &= \cancel{b^{\dagger}a^{\dagger}[b,b]} + \cancel{b^{\dagger}[b,a^{\dagger}]b} + a^{\dagger}[b^{\dagger},b]b + \cancel{[b^{\dagger},a^{\dagger}]bb} = -a^{\dagger}b\\
  \left[b^{\dagger}b, b^{\dagger}a\right] &= \cancel{b^{\dagger}b^{\dagger}[b,a]} + b^{\dagger}[b,b^{\dagger}]a + \cancel{b^{\dagger}[b^{\dagger},a]b} + \cancel{[b^{\dagger},b^{\dagger}]ab} = b^{\dagger}a\\
  \Rightarrow S^{y} &= \frac{1}{4i}\left(a^{\dagger}b - b^{\dagger}a + a^{\dagger}b - b^{\dagger}a\right) = \boxed{\frac{1}{2i}\left(a^{\dagger}b - b^{\dagger}a\right)}
\end{align*}}}

  \item \textbf{Prove that $S^{y}$ is actually related to the angular momentum operator of the harmonic oscillator $L = xp_{y} - yp_{x}$, namely $\begin{aligned}
    S^{y} = \frac{L}{2}
  \end{aligned}$.}
  
{\color{gray}{  Define 
  \begin{align*}
    x &= \frac{a + a^{\dagger}}{\sqrt{2}},\quad p_{x} = \frac{i(a^{\dagger} - a)}{\sqrt{2}}\\
    y &= \frac{b + b^{\dagger}}{\sqrt{2}},\quad p_{y} = \frac{i(b^{\dagger} - b)}{\sqrt{2}}
  \end{align*}
  So the angular momentum operator is
  \begin{align*}
    L &= \left(\frac{a + a^{\dagger}}{\sqrt{2}}\right)\left(\frac{i(b^{\dagger} - b)}{\sqrt{2}}\right) - \left(\frac{b + b^{\dagger}}{\sqrt{2}}\right)\left(\frac{i(a^{\dagger} - a)}{\sqrt{2}}\right)\\
    &= \frac{i}{2}\left[\left(a + a^{\dagger}\right)\left(b^{\dagger} - b\right) - \left(b + b^{\dagger}\right)\left(a^{\dagger} - a\right)\right]\\
    &= \frac{i}{2}\left(ab^{\dagger} - \cancel{ab} + \cancel{a^{\dagger}b^{\dagger}} - a^{\dagger}b - ba^{\dagger} + \cancel{ba} - \cancel{b^{\dagger}a^{\dagger}} + b^{\dagger}a\right)
  \end{align*}
  Because $[a,b] = [a,b^{\dagger}] = 0$, we have $ab^{\dagger} = b^{\dagger}a$ and $a^{\dagger}b = b a^{\dagger}$, so
  \begin{align*}
    L &= \frac{i}{2} \left(ab^{\dagger} - a^{\dagger}b - a^{\dagger}b + ab^{\dagger}\right) = i(ab^{\dagger}-a^{\dagger}b)
  \end{align*}
  While $\begin{aligned}
    S^{y} = \frac{1}{2i}(a^{\dagger}b - ab^{\dagger})= \frac{i}{2}(ab^{\dagger} - a^{\dagger}b)
  \end{aligned}$, so $\begin{aligned}
    S^{y} = \frac{L}{2}
  \end{aligned}$. }}$\square$
  
  \item \textbf{Define the following set of states, where $s = 0, 1/2, 1, \cdots$, and $m = -s, -s+1,\cdots, s-1, s$
  (they are called the Schwinger boson representation),
  \begin{align*}
    |s,m\rangle = \frac{(a^{\dagger})^{s+m}}{\sqrt{(s+m)!}}\frac{(b^{\dagger})^{s-m}}{\sqrt{(s-m)!}}|\Omega\rangle
  \end{align*}
  where $|\Omega\rangle$ is the state annihilated by $a$ and $b$, i.e., $a|\Omega\rangle = b|\Omega\rangle = 0$. Prove that the state $|s, m\rangle$ is indeed a simultaneous eigenstate of $\vec{S}^{2} = (S^{x})^{2} + (S^{y})^{2} + (S^{z})^{2}$ and $S^{z}$ , with eigenvalues $s(s + 1)$ and $m$ respectively. [Hint: Use the particle number basis.]}

  {\color{gray}{We have known that
  \begin{align*}
    S^{z} &= \frac{1}{2}\left(a^{\dagger}a - b^{\dagger}b\right)\\
    \vec{S}^{2} &= (S^{x})^{2} + (S^{y})^{2} + (S^{z})^{2}
  \end{align*}
  where $a^{\dagger}a$ counts the number of particles in the $a$ mode, and $b^{\dagger}b$ counts the number of particles in the $b$ mode. So we have
  \begin{align*}
    a^{\dagger}a|s,m\rangle = (s + m)|s,m\rangle,\quad b^{\dagger}b|s,m\rangle = (s - m)|s,m\rangle\\
    \Rightarrow S^{z}|s,m\rangle = \frac{1}{2}\left((s + m) - (s - m)\right)|s,m\rangle = \boxed{m|s,m\rangle}
  \end{align*}
  So $|s,m\rangle$ is an eigenstate of $S^{z}$ with eigenvalue $m$.

  Define ladder operators $S^{\pm} = S^{x} \pm iS^{y}$:
  \begin{align*}
    S^{+} &= a^{\dagger}b, \quad S^{-} = b^{\dagger}a\\
    \Rightarrow S^{2} &= S^{z}S^{z} + \frac{1}{2}\left(S^{+}S^{-} + S^{-}S^{+}\right)
  \end{align*}
  接下来证明 Schwinger boson 表象下定义的态 $|s,m\rangle$ 以及对应的升降算符 $S^{\pm}$ 仍然满足传统的波函数关系. 以 $S^{+} = a^{\dagger}b$ 为例:
  \begin{align*}
    S^{+}|s,m\rangle &= a^{\dagger}b
    \frac{(a^{\dagger})^{s+m}}{\sqrt{(s+m)!}}
    \frac{(b^{\dagger})^{s-m}}{\sqrt{(s-m)!}}
    |\Omega\rangle \\
    &= \frac{\sqrt{s+m+1}}{\sqrt{s-m}}
    \frac{(a^{\dagger})^{s+m+1}}{\sqrt{(s+m+1)!}}
    bb^{\dagger}
    \frac{(b^{\dagger})^{s-m-1}}{\sqrt{(s-m-1)!}}
    |\Omega\rangle \\
    &= \frac{\sqrt{s+m+1}}{\sqrt{s-m}}(b^{\dagger}b + 1)|s,m+1\rangle\\
    &= \frac{\sqrt{s+m+1}}{\sqrt{s-m}}(s-m-1 + 1)|s,m+1\rangle\\
    &= \sqrt{s(s+1)-m(m+1)}|s,m+1\rangle
  \end{align*}
  说明该定义下的算符仍然满足传统的数值关系, $S^{-}$ 证明略. 则我们有
  \begin{align*}
    S^{+}|s,m\rangle &= a^{\dagger}b|s,m\rangle = \sqrt{s(s+1)-m(m+1)}|s,m+1\rangle\\
    S^{-}|s,m\rangle &= b^{\dagger}a|s,m\rangle = \sqrt{s(s+1)-m(m-1)}|s,m-1\rangle\\
    \Rightarrow 
    S^{+}S^{-}|s,m\rangle &= S^{+}\sqrt{s(s+1)-m(m-1)}|s,m-1\rangle = \bigg[s(s+1)-m(m-1)\bigg]|s,m\rangle\\
    S^{-}S^{+}|s,m\rangle &= S^{-}\sqrt{s(s+1)-m(m+1)}|s,m+1\rangle = \bigg[s(s+1)-m(m+1)\bigg]|s,m\rangle\\
    S^{z}S^{z}|s,m\rangle &= m^{2}|s,m\rangle
  \end{align*}
  Combine the above results, and we have
  \begin{align*}
    S^{2}|s,m\rangle &= S^{z}S^{z}|s,m\rangle + \frac{1}{2}\left(S^{+}S^{-} + S^{-}S^{+}\right)|s,m\rangle\\
    &= m^{2}|s,m\rangle + \frac{1}{2}\bigg[s(s+1)-m(m-1) + s(s+1)-m(m+1)\bigg]|s,m\rangle\\
    &= \boxed{s(s+1)|s,m\rangle}
  \end{align*}}}
  $\square$
\end{enumerate}

\subsection{1D tight-binding model}
\textbf{The Hamiltonian of a periodic tight-binding chain of length $L$ is given by the following expression:%长度为 $L$ 的周期性紧密结合链的哈密顿方程由以下表达式给出:
\begin{align*}
  H_{\text{chain}} = -t\sum_{n = 1}^{L}\left(\hat{a}_{n}^{\dagger}\hat{a}_{n+1} + \hat{a}_{n+1}^{\dagger}\hat{a}_{n}\right)
\end{align*}
where $t$ is the hopping matrix element between adjacent sites $n$ and $n + 1$, $\hat{a}_{n}^{\dagger}$ creates a fermion at site $n$, and the set of operators $\{a_{n}^{\dagger}, a_{n}; n = 1,\cdots, L\}$ satisfies the standard anticommutation relations: %其中 $t$ 是相邻位点 $n$ 和 $n + 1$ 之间的跳跃矩阵元素,$\hat{a}_{n}^{\dagger}$ 在位点 $n$ 创建费米子,算子集 $\{a_{n}^{\dagger}, a_{n}; n = 1,\cdots, L\}$ 满足标准的反对易关系
\begin{align*}
  \{a_{n},a_{n^{\prime}}^{\dagger}\} = \delta_{nn^{\prime}},\quad \{a_{n},a_{n^{\prime}}\} = 0,\quad \{a_{n}^{\dagger},a_{n^{\prime}}^{\dagger}\} = 0
\end{align*}
We assume periodic boundary conditions, i.e., we consider $a_{L+n}^{\dagger} = a_{n}^{\dagger}$.%我们假设周期性边界条件,即考虑 $a_{L+n}^{\dagger} = a_{n}^{\dagger}$。
The purpose of this problem is to prove that this Hamiltonian can be diagonalized by a linear transformation of the discrete Fourier transform form:%这个问题的目的是证明,通过离散傅里叶变换形式的线性变换,可以将这个哈密顿对角化:
\begin{align*}
  b^{\dagger}_{k} = \frac{1}{\sqrt{L}}\sum_{n = 1}^{L}e^{ikn}a_{n}^{\dagger}
\end{align*}}

\begin{enumerate}
  \item \textbf{Let's require that $b^{\dagger}_{k}$ remains invariant under any shift of the summation index $n\rightarrow n + n^{\prime}$ ("translation invariance"). Prove that this implies that the index $k$ is quantized and determine the set of allowed $k$ values. How many independent $b^{\dagger}_{k}$ operators are there?} 
  
  {\color{gray}{不妨令 $n\rightarrow n + 1$, 有
  \begin{align*}
    b^{\dagger}_{k} &= \frac{1}{\sqrt{L}}\sum_{n = 1}^{L} e^{ik(n + 1)}a_{n + 1}^{\dagger} = \frac{1}{\sqrt{L}}\sum_{n^{\prime} = 2}^{L+1} e^{ikn^{\prime}}a_{n^{\prime}}^{\dagger}\\
    &= \frac{1}{\sqrt{L}}\left[\sum_{n^{\prime} = 2}^{L} e^{ikn^{\prime}}a_{n^{\prime}}^{\dagger} + e^{ik(L+1)}a^{\dagger}_{L+1}\right]\\
    &= \frac{1}{\sqrt{L}}\left[\sum_{n^{\prime} = 1}^{L} e^{ikn^{\prime}}a_{n^{\prime}}^{\dagger} - e^{ik}a_{1}^{\dagger} + e^{ik(L+1)}a^{\dagger}_{L+1}\right]\\
    \Rightarrow e^{ik}a_{1}^{\dagger} &= e^{ik(L+1)}a^{\dagger}_{L+1} = e^{ik(L+1)}a^{\dagger}_{1}\\
    \Rightarrow e^{ikL} &= 1 = e^{i2\pi  m},\quad m\in\mathbb{Z}\\
    \Rightarrow k &= \frac{2\pi}{L}m,\quad m\in\{0,1,2,\cdots,L-1\}
  \end{align*}
  So there are $\boxed{L}$ independent $b^{\dagger}_{k}$ operators.}}
  
  \item \textbf{Verify that the set of $b_{k}$ and $b^{\dagger}_{k}$ operators also satisfies the above standard anticommutation relations. That is:
  \begin{align*}
    \{b_{k},b_{k^{\prime}}^{\dagger}\} = \delta_{kk^{\prime}},\quad \{b_{k},b_{k^{\prime}}\} = 0,\quad \{b_{k}^{\dagger},b_{k^{\prime}}^{\dagger}\} = 0
  \end{align*}
  Hint: Use the identity $\begin{aligned}
    \sum_{m=1}^{L} e^{i\frac{2\pi}{L}m}= 0
  \end{aligned}$.}

{\color{gray}{  We have
  \begin{align*}
    b^{\dagger}_{k} = \frac{1}{\sqrt{L}}\sum_{n = 1}^{L}e^{ikn}a_{n}^{\dagger},\quad b_{k} = \frac{1}{\sqrt{L}}\sum_{n = 1}^{L}e^{-ikn}a_{n}
  \end{align*}
  So 
  \begin{align*}
    \{b_{k},b^{\dagger}_{k^{\prime}}\} &= \frac{1}{L}\sum_{n,n^{\prime}} e^{-ikn}e^{ik^{\prime}n^{\prime}}\{a_{n},a_{n^{\prime}}^{\dagger}\} = \frac{1}{L}\sum_{n,n^{\prime}} e^{-ikn}e^{ik^{\prime}n^{\prime}}\delta_{nn^{\prime}}\\
    &= \frac{1}{L}\sum_{n=1}^{L} e^{-ikn}e^{ik^{\prime}n} = \frac{1}{L}\sum_{n =1}^{L} e^{i(k^{\prime} - k)n} = \boxed{\delta_{kk^{\prime}}}\\
    \{b_{k},b_{k^{\prime}}\} &= \frac{1}{L}\sum_{n,n^{\prime}} e^{-ikn}e^{-ik^{\prime}n^{\prime}}\{a_{n},a_{n^{\prime}}\} = \boxed{0}\\
    \{b_{k}^{\dagger},b_{k^{\prime}}^{\dagger}\} &= \frac{1}{L}\sum_{n,n^{\prime}} e^{ikn}e^{ik^{\prime}n^{\prime}}\{a_{n}^{\dagger},a_{n^{\prime}}^{\dagger}\} = \boxed{0}
  \end{align*}}}

  \item \textbf{Prove that the inverse transformation of the above has the form:
  \begin{align*}
    a_{n}^{\dagger} = \frac{1}{\sqrt{L}}\sum_{k}e^{-ikn}b_{k}^{\dagger}
  \end{align*}
  where the sum is over the set of allowed $k$ values determined in (a).}

{\color{gray}{  We have the definition
  \begin{align*}
    b^{\dagger}_{k} = \frac{1}{\sqrt{L}}\sum_{n = 1}^{L}e^{ikn}a_{n}^{\dagger}
  \end{align*}
  So 
  \begin{align*}
    \frac{1}{\sqrt{L}}\sum_{k}e^{-ikn}b_{k}^{\dagger} &= \frac{1}{\sqrt{L}}\sum_{k}e^{-ikn}\left(\frac{1}{\sqrt{L}}\sum_{n^{\prime}}e^{ikn^{\prime}}a_{n^{\prime}}^{\dagger}\right)\\
    &= \frac{1}{L}\sum_{n^{\prime}}\sum_{k}e^{ik(n^{\prime}-n)}a_{n^{\prime}}^{\dagger} = \sum_{n^{\prime}}\left(\frac{1}{L}\sum_{k}e^{ik(n^{\prime}-n)}\right)a_{n^{\prime}}^{\dagger}\\
    &= \sum_{n^{\prime}}(\delta_{nn^{\prime}})a_{n^{\prime}}^{\dagger} = a_{n}^{\dagger}.\quad \square
  \end{align*}}}

  \item \textbf{Show that $b^{\dagger}_{k}$ is indeed a creation operator of a single-particle eigenstate of $H_{\text{chain}}$ by proving that its commutator with the Hamiltonian has the form $[H_{\text{chain}}, b^{\dagger}_{k}] = \varepsilon_{k}b^{\dagger}_{k}$. Give the explicit expression for the corresponding eigenvalue $\varepsilon_{k}$.}
  
{\color{gray}{  We have known that
  \begin{align*}
    H_{\text{chain}} &= -t\sum_{n = 1}^{L}\left(\hat{a}_{n}^{\dagger}\hat{a}_{n+1} + \hat{a}_{n+1}^{\dagger}\hat{a}\right),\quad \hat{a}_{L+1} = \hat{a}_{1}\\
    b^{\dagger}_{k} &= \frac{1}{\sqrt{L}}\sum_{n = 1}^{L}e^{ikn}a_{n}^{\dagger}
  \end{align*}
  So the commutator
  \begin{align*}
    [H_{\text{chain}},b^{\dagger}_{k}] &= -t\sum_{n = 1}^{L}\left(\left[a_{n}^{\dagger}a_{n+1},b^{\dagger}_{k}\right] + \left[a_{n+1}^{\dagger}a_{n},b^{\dagger}_{k}\right]\right)\\
    &= -\frac{t}{L}\sum_{n=1}^{L}\sum_{n^{\prime}}^{L}\left(\left[
      a_{n}^{\dagger}a_{n+1},e^{ikn^{\prime}}a_{n^{\prime}}^{\dagger}\right] + \left[a_{n+1}^{\dagger}a_{n},e^{ikn^{\prime}}a_{n^{\prime}}^{\dagger}\right]\right)\\
    &= -\frac{t}{L}\sum_{n=1}^{L}\sum_{n^{\prime}}^{L}e^{ikn^{\prime}}\left(
      a_{n}^{\dagger}a_{n+1}a_{n^{\prime}}^{\dagger} 
      - \underbrace{a_{n^{\prime}}^{\dagger}a_{n}^{\dagger}a_{n+1}}_{*} 
      + a_{n+1}^{\dagger}a_{n}a_{n^{\prime}}^{\dagger} 
      - \underbrace{a_{n^{\prime}}^{\dagger}a_{n+1}^{\dagger}a_{n}}_{*}
      \right)
  \end{align*}
根据 $a,a^{\dagger}$ 的反对易关系, 交换相邻的升算符和降算符满足关系 $\begin{aligned}
  \left\{\begin{aligned}
    a^{\dagger}_{n^{\prime}}a_{n}^{\dagger} &= -a_{n}^{\dagger}a_{n^{\prime}}^{\dagger}\\
    a_{n^{\prime}}a_{n} &= -a_{n}a_{n^{\prime}}\\
  \end{aligned}\right.
\end{aligned}$
交换 * 项中的升算符, 从而使其变号:
\begin{align*}
  [H_{\text{chain}},b^{\dagger}_{k}] &= -\frac{t}{\sqrt{L}}\sum_{n=1}^{L}\sum_{n^{\prime}}^{L}e^{ikn^{\prime}}\left(
    a_{n}^{\dagger}a_{n+1}a_{n^{\prime}}^{\dagger} 
    + a_{n}^{\dagger}a_{n^{\prime}}^{\dagger}a_{n+1} 
    + a_{n+1}^{\dagger}a_{n}a_{n^{\prime}}^{\dagger} 
    + a_{n+1}^{\dagger}a_{n^{\prime}}^{\dagger}a_{n}
    \right)\\
    &= -\frac{t}{\sqrt{L}}\sum_{n=1}^{L}\sum_{n^{\prime}}^{L}e^{ikn^{\prime}}\left[
      a_{n}^{\dagger} \underbrace{\left(a_{n+1}a_{n^{\prime}}^{\dagger} + a_{n^{\prime}}^{\dagger}a_{n+1}\right)}_{\{a_{n+1},a_{n^{\prime}}^{\dagger}\}}
      + a_{n+1}^{\dagger} \underbrace{\left( a_{n}a_{n^{\prime}}^{\dagger} + a_{n^{\prime}}^{\dagger}a_{n} \right)}_{\{a_{n},a_{n^{\prime}}^{\dagger}\}}
      \right]\\
    &= -\frac{t}{\sqrt{L}}\sum_{n=1}^{L}\sum_{n^{\prime}}^{L}\left[
      e^{ikn^{\prime}}a_{n}^{\dagger}\delta_{n+1,n^{\prime}}
      + e^{ikn^{\prime}}a_{n+1}^{\dagger}\delta_{n,n^{\prime}}
      \right]\\
      &= -\frac{t}{\sqrt{L}}\sum_{n=1}^{L}\left[
      e^{ik}e^{ikn}a^{\dagger}_{n} + e^{-ik}e^{ik(n+1)}a_{n+1}^{\dagger}
      \right]\\
      &= -t\left[
        e^{ik}b^{\dagger}_{k} + e^{-ik}b^{\dagger}_{k}
      \right]\\
      \varepsilon_{k}b_{k}^{\dagger} &= -2t\cos{k}b_{k}^{\dagger}
\end{align*}

So the corresponding eigenvalue $\boxed{\varepsilon_{k} = -2t\cos{k}}$.}}
\end{enumerate}
\end{document}