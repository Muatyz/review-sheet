根据对角分解有 $H = c^{\dagger}VDV^{-1}c$, 合并 $V^{-1}c$ 为 $\gamma$, 即得到矩阵的新基矢为 $\gamma\equiv V^{-1}c$. 同样的, $c = V\gamma$, 或者写作求和约定 $\begin{aligned}
    c_{\alpha} = \sum_{i}V_{\alpha i}\gamma_{i}
  \end{aligned}$. 基态被定义为占据最低能量的态, 而根据对角矩阵可以发现最低能量是二重简并的, 是新基矢 $\gamma$ 的第 $1,2$ 分量给出的, 因此基态使用产生算符 $\times|0\rangle$ 写出的话将会是$\begin{aligned}
    \prod_{\varepsilon_{i}<\varepsilon_{F}}\gamma_{i}^{\dagger}|0\rangle = \gamma_{1}^{\dagger}\gamma_{2}^{\dagger}|0\rangle
  \end{aligned}$. 那么各粒子数平均值为 
  \begin{align*}
    \langle n_{1\uparrow}\rangle 
    &= \langle c_{1\uparrow}^{\dagger}c_{1\uparrow}\rangle
     = \sum_{i,j}(V_{1\uparrow,i})^{\dagger}V_{1\uparrow,j}\langle\gamma_{i}^{\dagger}\gamma_{j}\rangle\\
    &= \sum_{i,j}(V_{1\uparrow, i})^{\dagger}V_{1\uparrow,j}\delta_{ij} 
     = \sum_{i}(V_{1\uparrow, i})^{\dagger}V_{1\uparrow,i} 
     = (V_{1\uparrow,1})^{\dagger}V_{1\uparrow,1} + (V_{1\uparrow,2})^{\dagger}V_{1\uparrow,2} \\
    &= \frac{1}{2}
  \end{align*}
  同理计算得到 $\begin{aligned}
    \langle n_{1\downarrow}\rangle = \langle n_{2\uparrow}\rangle = \langle n_{2\downarrow}\rangle = \frac{1}{2}
  \end{aligned}$. 这是顺磁态, 能量为 
  \begin{align*}
    E_{\text{HF}} &= \sum_{\varepsilon_{\alpha}<0}\varepsilon_{\alpha}
    - U\cdot \frac{1}{2}\frac{1}{2}\times 2 = \left(-t + \frac{U}{2}\right)\times 2 - \frac{U}{2}\\
    &= -2t + \frac{U}{2}
  \end{align*}