\documentclass[../../main.tex]{subfiles}
\graphicspath{{\subfix{../images/}}} % 指定图片目录,后续可以直接使用图片文件名。
\begin{document}
\section{量子计算基础}
\subsection{量子纠缠}
\subsubsection{双量子比特态}
量子比特有两种状态 $|\uparrow\rangle = \begin{pmatrix}
    1 \\ 0
\end{pmatrix}$ 和 $|\downarrow\rangle = \begin{pmatrix}
    0 \\ 1
\end{pmatrix}$. 通过张量积规则 $\begin{aligned}
    \begin{pmatrix}
        a_{1} \\ a_{2}
    \end{pmatrix}\otimes \begin{pmatrix}
        b_{1} \\ b_{2}
    \end{pmatrix} = \begin{pmatrix}
        a_{1}\begin{pmatrix}
            b_{1} \\ b_{2}
        \end{pmatrix} \\ a_{2}\begin{pmatrix}
            b_{1} \\ b_{2}
        \end{pmatrix}
    \end{pmatrix} = \begin{pmatrix}
        a_{1}b_{1} \\ a_{1}b_{2} \\ a_{2}b_{1} \\ a_{2}b_{2}
    \end{pmatrix}
\end{aligned}$ 
计算复合系统的基矢 $|\uparrow\uparrow\rangle$, 
$|\uparrow\downarrow\rangle$, $|\downarrow\uparrow\rangle$, $|\downarrow\downarrow\rangle$. 所以双量子比特 Hilbert 空间中的态可以展开为基矢的线性组合:

\begin{align*}
    |\psi\rangle = \psi_{1}|\uparrow\uparrow\rangle + \psi_{2}|\uparrow\downarrow\rangle + \psi_{3}|\downarrow\uparrow\rangle + \psi_{4}|\downarrow\downarrow\rangle = \begin{pmatrix}
        \psi_{1} \\ \psi_{2} \\ \psi_{3} \\ \psi_{4}
    \end{pmatrix}
\end{align*}

\subsubsection{双量子比特算符}

通过 Pauli 矩阵约定 $\begin{aligned}
    \sigma_{0} = \begin{pmatrix}
        1 & 0 \\ 0 & 1
    \end{pmatrix}
\end{aligned}$, $\sigma^{1,2,3}=\sigma^{x,y,z}$, 且其张量积积简写为 $\begin{aligned}
    \sigma_{A}^{i}\otimes\sigma_{B}^{j}\equiv \sigma^{ij}
\end{aligned}$, 矩阵张量积规则为

\begin{align*}
    \sigma^{32} = \begin{pmatrix}
        1 & 0 \\ 0 & -1
    \end{pmatrix}\otimes \sigma^{2} = \begin{pmatrix}
1\sigma^{2} & 0\sigma^{2} \\ 0\sigma^{2} & -1\sigma^{2}
    \end{pmatrix} = \begin{pmatrix}
        & -i & & \\
        i &  &  & \\
        &  &  & i \\
        &  & -i &
    \end{pmatrix}
\end{align*}

这相当于是在给定算符的 "基". 即观测量矩阵都可以展开为这些矩阵张量积的线性组合. 谈论单量子比特的观测量时, 相当于默认另一个量子比特算符为 $\mathbb{I} = \sigma^{0}$, 使得算符基为 $(\sigma^{10},\sigma^{20},\sigma^{30})$ 和 $(\sigma^{01},\sigma^{02},\sigma^{03})$.

\subsubsection{双量子比特模型}

双量子比特 Heisenberg 模型哈密顿量为 $\begin{aligned}
    H = \frac{J}{4}\vec{\sigma}_{A}\cdot\vec{\sigma}_{B}
\end{aligned}$, 将其写作矩阵形式:
\begin{align*}
    H = \frac{J}{4}(\sigma^{11}+\sigma^{22}+\sigma^{33}) = \frac{J}{4} \begin{pmatrix}
        1 &  & & \\
         & -1 & 2 & \\
         & 2 & -1 & \\
            &  &  & 1
    \end{pmatrix}
\end{align*}

将其对角化以计算本征值, 并找到本征值对应的本征态, 结果为
\begin{align*}
    E_{s} &= -\frac{3}{4}J,\quad |s\rangle = \frac{1}{\sqrt{2}}(|\uparrow\downarrow\rangle - |\downarrow\uparrow\rangle),\quad \text{自旋单态}\\
    E_{t} &= \frac{J}{4},\quad \left\{\begin{aligned}
        |t_{1}\rangle &= \frac{1}{\sqrt{2}}(|\uparrow\downarrow\rangle + |\downarrow\uparrow\rangle)\\
        |t_{2}\rangle &= \frac{1}{\sqrt{2}}(|\uparrow\uparrow\rangle + |\downarrow\downarrow\rangle)\\
        |t_{3}\rangle &= \frac{1}{\sqrt{2}}(|\uparrow\uparrow\rangle - |\downarrow\downarrow\rangle)
    \end{aligned}\right.,\quad \text{自旋三重态}
\end{align*}

\subsubsection{自旋单态}

\begin{align*}
    |s\rangle = \frac{1}{\sqrt{2}}(|\uparrow\downarrow\rangle - |\downarrow\uparrow\rangle) = \frac{1}{\sqrt{2}}\begin{pmatrix}
        0 \\ 1 \\ -1 \\ 0
    \end{pmatrix},\vec{\sigma}_{A} = (\sigma^{10},\sigma^{20},\sigma^{30}) = \left(\begin{pmatrix}
         &  & 1 &  \\  &  &  & 1\\ 1 &  &  &  \\  & 1 &  & 
    \end{pmatrix}, \begin{pmatrix}
         &  & -i &  \\  &  &  & -i\\ i &  &  &  \\  & i &  & 
    \end{pmatrix}, \begin{pmatrix}
        1 &  &  &  \\  & 1 &  & \\  &  & -1 &  \\  &  &  & -1
    \end{pmatrix}
    \right)
\end{align*}

\subsubsection{纠缠熵}
双量子比特态 $|\psi\rangle$ 中量子比特 $A$ 的纠缠熵: $\begin{aligned}
    S(A) = -\text{Tr}[\rho_{A}\ln{\rho_{A}}]
\end{aligned}$. 其中约化密度矩阵 $\begin{aligned}
    \rho_{A} = \text{Tr}_{B}|\psi\rangle\langle\psi|
\end{aligned}$

更广义的 Renyi 纠缠熵 $\begin{aligned}
    S^{(n)}(A) = \frac{1}{1-n}\ln{\text{Tr}\rho_{A}^{n}}
\end{aligned}$. 

接下来介绍如何部分求迹. 

\begin{enumerate}
    \item 自旋单态 $\begin{aligned}
        |\psi\rangle = |s\rangle = \frac{1}{\sqrt{2}}(|\uparrow\downarrow\rangle - |\downarrow\uparrow\rangle)
    \end{aligned}$. 其总密度矩阵为
    \begin{align*}
        \rho &= |s\rangle\langle s| = \frac{1}{2}\begin{pmatrix}
            0 \\ 1 \\ -1 \\ 0
        \end{pmatrix}\begin{pmatrix}
            0 & 1 & -1 & 0
        \end{pmatrix} = \frac{1}{2}\begin{pmatrix}
             & & & \\
                & 1 & -1 & \\
                & -1 & 1 & \\
                & & &
        \end{pmatrix}\\
        \rho_{A} &= \text{Tr}_{B}\rho = \frac{1}{2}\begin{pmatrix}
            \text{Tr}\begin{pmatrix}
                & \\ & 1
            \end{pmatrix} & \text{Tr}\begin{pmatrix}
                & \\ -1 &
            \end{pmatrix} \\ \text{Tr}\begin{pmatrix}
                 & -1 \\ & 
            \end{pmatrix} & \text{Tr}\begin{pmatrix}
                1 & \\ &
            \end{pmatrix}
        \end{pmatrix} = \frac{1}{2}\begin{pmatrix}
            1 & 0 \\ 0 & 1
        \end{pmatrix},\quad \lambda_{A}^{1} = \frac{1}{2}, \quad \lambda_{A}^{2} = \frac{1}{2}\\
        S(A) &= -\text{Tr}\rho_{A}\ln{\rho_{A}} = - \sum_{i}\lambda_{A}^{i}\ln{\lambda_{A}^{i}} = -\left(\frac{1}{2}\ln{\frac{1}{2}} + \frac{1}{2}\ln{\frac{1}{2}}\right) = \ln{2}
    \end{align*}
    \item 乘积态 $\begin{aligned}
        |\psi\rangle = \frac{1}{2}(|\uparrow\uparrow\rangle + |\uparrow\downarrow\rangle + |\downarrow\uparrow\rangle + |\downarrow\downarrow\rangle)
    \end{aligned}$. 总密度矩阵为
    \begin{align*}
        \rho &= |\psi\rangle\langle\psi| = \frac{1}{4}\begin{pmatrix}
            1 \\ 1 \\ 1 \\ 1
        \end{pmatrix}\begin{pmatrix}
            1 & 1 & 1 & 1
        \end{pmatrix} = \frac{1}{4}\begin{pmatrix}
            1 & 1 & 1 & 1\\ 1 & 1 & 1 & 1\\ 1 & 1 & 1 & 1\\ 1 & 1 & 1 & 1
        \end{pmatrix}\\
        \rho_{A} &= \text{Tr}_{B}\rho = \frac{1}{4}\begin{pmatrix}
            \text{Tr}\begin{pmatrix}
                1 & 1 \\ 1 & 1
            \end{pmatrix} &  \text{Tr}\begin{pmatrix}
                1 & 1 \\ 1 & 1
            \end{pmatrix}\\
            \text{Tr}\begin{pmatrix}
                1 & 1 \\ 1 & 1
            \end{pmatrix} &  \text{Tr}\begin{pmatrix}
                1 & 1 \\ 1 & 1
            \end{pmatrix}
        \end{pmatrix} = \frac{1}{4}\begin{pmatrix}
            2 & 2 \\ 2 & 2
        \end{pmatrix} = \frac{1}{2}\begin{pmatrix}
            1 & 1 \\ 1 & 1
        \end{pmatrix},\quad \lambda_{A}^{1} = 0,\quad \lambda_{A}^{2} = 1\\
        S(A) &= -\text{Tr}\rho_{A}\ln{\rho_{A}} = -\sum_{i}\lambda_{A}^{i}\ln{\lambda_{A}^{i}} = - (0\ln{0} + 1\ln{1}) = 0.
    \end{align*}
\end{enumerate}

\subsubsection{互信息}
双量子比特体系, A 和 B 之间的互信息为

\begin{align*}
    I(A:B) = S(A) + S(B) - S(A\cup B)
\end{align*}

\subsubsection{EPR 佯谬和 Bell 不等式}

\end{document}