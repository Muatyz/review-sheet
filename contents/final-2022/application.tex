\documentclass[../../main.tex]{subfiles}
\graphicspath{{\subfix{../images/}}} % 指定图片目录,后续可以直接使用图片文件名。
\begin{document}
\section{应用题}
\begin{enumerate}
  \item \textbf{矩阵对角化和表象变换}
  \begin{enumerate}
    \item \textbf{对角化矩阵 $L$ 就是去找到幺正变换 $V$, 使得 $L = V\Lambda V^{\dagger}$, 其中 $\Lambda$ 是一个对角矩阵, 它的对角元是本征值. $V$ 是一个幺正矩阵, 它的列矢量是本征矢, 和 $\Lambda$ 中的本征值一一对应. 找到一个能对角化 Pauli 矩阵 $\sigma^{x} = \begin{pmatrix}
      0 & 1\\
      1 & 0
    \end{pmatrix}$ 的幺正矩阵 $V$, 并找到 $\sigma^{x}$ 的本征值. }

    {\color{gray}{通过求解其特征方程以得到 $\sigma^{x}_{(z)}$ 的本征值:
    \begin{align*}
      \text{det}(\sigma^{x}_{(z)}-\lambda I) = \text{det}\begin{pmatrix}
        -\lambda & 1 \\
        1 & -\lambda
      \end{pmatrix} = \lambda^{2} - 1 = 0,
    \end{align*}
    解得 $\boxed{\lambda = \pm 1}$. 对于 $\lambda_{+} = 1$ 有: 
    \begin{align*}
      \begin{pmatrix}
        0 & 1 \\
        1 & 0
      \end{pmatrix}
      \begin{pmatrix}
        v_{1}\\
        v_{2}
      \end{pmatrix}
      = 1\cdot\begin{pmatrix}
        v_{1}\\
        v_{2}
      \end{pmatrix}\Rightarrow v_{1} = v_{2}.
    \end{align*}
    所以对应于 $\lambda_{+}$ 的本征矢是 $\begin{aligned}
      |+\rangle^{x}_{(z)} = \frac{1}{\sqrt{2}}\begin{pmatrix}
        1\\
        1
      \end{pmatrix}
    \end{aligned}$. 对于 $\lambda_{-} = -1$ 有
    \begin{align*}
      \begin{pmatrix}
        0 & 1 \\
        1 & 0
      \end{pmatrix}
      \begin{pmatrix}
        v_{1}\\
        v_{2}
      \end{pmatrix}
      = -1\cdot\begin{pmatrix}
        v_{1}\\
        v_{2}
      \end{pmatrix}\Rightarrow v_{1} = -v_{2}.
    \end{align*}
    所以对应于 $\lambda_{-}$ 的本征矢是 $\begin{aligned}
      |-\rangle^{x}_{(z)} = \frac{1}{\sqrt{2}}\begin{pmatrix}
        1\\
        -1
      \end{pmatrix}
    \end{aligned}$. 在求解过程中已经对这些本征矢进行了归一化, 所以可以得到幺正矩阵 $\begin{aligned}
      V = [|+\rangle^{x}_{(z)},|-\rangle^{x}_{(z)}] = \frac{1}{\sqrt{2}}\begin{pmatrix}
        1 & 1 \\
        1 & -1
      \end{pmatrix}
    \end{aligned}$.
    对角矩阵 $\Lambda$ 对角线上依次是本征值, 即
    \begin{align*}
      \Lambda = \text{diag}\{\lambda_{+},\lambda_{-}\} =\begin{pmatrix}
        1 & 0 \\
        0 & -1
      \end{pmatrix} = \sigma^{z}_{(z)}
    \end{align*}
    于是我们可以通过幺正矩阵 $V$ 来对 $\sigma^{x}_{(z)}$ 进行对角化: 
    \begin{align*}
      \sigma^{x}_{(z)} = V^{\dagger}\Lambda V = V^{\dagger}\sigma^{z}_{(z)} V
    \end{align*}
    我们注意到, 对角矩阵 $\Lambda$ 和 $\sigma^{z}_{(z)}$ 形式完全一致, 这意味着不同表象 $i$ 下, $\sigma^{i}_{(i)}$ 的形式都是 $\begin{aligned}
      \begin{pmatrix}
        1 & 0 \\
        0 & -1
      \end{pmatrix}
    \end{aligned}$, 这就是我们通过 $V$ 来改变表象的依据: 
    \begin{align*}
      \sigma^{x}_{(z)} = V^{\dagger}\sigma^{z}_{(z)} V = V^{\dagger}\sigma^{x}_{(x)} V\Rightarrow \sigma^{x}_{(x)} = \left(V^{\dagger}\right)^{-1}\sigma^{x}_{(z)}(V)^{-1}
    \end{align*}
    我们标记 $\sigma^{x}_{(z)}$ 为 $\sigma^{x}$ 在 $\sigma^{z}$ 表象下的矩阵. 注意 $V = V^{\dagger} = V^{-1}$, 所以
    \begin{align*}
      \sigma^{x}_{(x)} = V\sigma^{x}_{(z)}V
    \end{align*}}}

    \item \textbf{自旋 $1/2$ 的自旋角动量算符 $\vec{S}$ 的三个分量为$S^{x}$, $S^{y}$, $S^{z}$. 如果采用 $S^{z}$ 表象, 它们的矩阵表示为 $\begin{aligned}
      \vec{S} = \frac{\hbar}{2}\vec{\sigma}
    \end{aligned}$, 其中 $\vec{\sigma}$ 的三个分量为 Pauli 矩阵 $\sigma^{x}$, $\sigma^{y}$, $\sigma^{z}$. 现在考采用 $S^{x}$ 表象, 请列出 $S^{x}$ 表象中你约定的基矢顺序, 并求出在该表象下算符 $\vec{S}$ 的三个分量的矩阵表示. }

    {\color{gray}{在 $S^{z}$ 表象下有
    \begin{align*}
      S^{x}_{(z)} = \frac{\hbar}{2}\sigma^{x}_{(z)} = \frac{\hbar}{2}\begin{pmatrix}
        0 & 1 \\
        1 & 0
      \end{pmatrix}
    \end{align*}
    从前文中可知, $\sigma^{x}_{(z)}$ 的本征矢为: 
    \begin{align*}
      |+\rangle_{x} = \frac{1}{\sqrt{2}}\begin{pmatrix}
        1\\
        1
      \end{pmatrix},\quad |-\rangle_{x} = \frac{1}{\sqrt{2}}\begin{pmatrix}
        1\\
        -1
      \end{pmatrix}.
    \end{align*}
    用以将 $S^{z}$ 表象转换为 $S^{x}$ 表象的幺正矩阵为
    \begin{align*}
      V = \frac{1}{\sqrt{2}}\begin{pmatrix}
        1 & 1 \\
        1 & -1
      \end{pmatrix}
    \end{align*}
    在 $S^{z}$ 表象中有
    \begin{align*}
      S^{x}_{(z)} = \frac{\hbar}{2}\sigma^{x} = \frac{\hbar}{2}\begin{pmatrix}
        0 & 1 \\
        1 & 0
      \end{pmatrix},\quad
      S^{y}_{(z)} = \frac{\hbar}{2}\sigma^{y} = \frac{\hbar}{2}\begin{pmatrix}
        0 & -i \\
        i & 0
      \end{pmatrix},\quad
      S^{z}_{(z)} = \frac{\hbar}{2}\sigma^{z} = \frac{\hbar}{2}\begin{pmatrix}
        1 & 0 \\
        0 & -1
      \end{pmatrix}.
    \end{align*}
    因此
    \begin{align*}
      S^{x}_{(x)} &= VS^{x}_{(z)}V = \frac{\hbar}{2}\begin{pmatrix}
        1&0\\
        0&-1
      \end{pmatrix},\\ 
      S^{y}_{(x)} &= VS^{y}_{(z)}V = \frac{1}{\sqrt{2}}\begin{pmatrix}
        1 & 1 \\
        1 & -1
      \end{pmatrix}\frac{\hbar}{2}\begin{pmatrix}
        0 & -i \\
        i & 0
      \end{pmatrix}\frac{1}{\sqrt{2}}\begin{pmatrix}
        1 & 1 \\
        1 & -1
      \end{pmatrix} = \frac{\hbar}{2}\begin{pmatrix}
        0 & -i \\
        i & 0
      \end{pmatrix},\\
      S^{z}_{(x)} &= VS^{z}_{(z)}V = \frac{1}{\sqrt{2}}\begin{pmatrix}
        1 & 1 \\
        1 & -1
      \end{pmatrix}\frac{\hbar}{2}\begin{pmatrix}
        1 & 0 \\
        0 & -1
      \end{pmatrix}\frac{1}{\sqrt{2}}\begin{pmatrix}
        1 & 1 \\
        1 & -1
      \end{pmatrix} = \frac{\hbar}{2}\begin{pmatrix}
        0 & 1\\
        1 & 0
      \end{pmatrix}.
    \end{align*}
    在 $S^{x}$ 表象中的基矢为 
    \begin{align*}
      |+\rangle_{(x)}^{x} = \begin{pmatrix}
        1\\
        0
      \end{pmatrix},\quad |-\rangle_{(x)}^{x} = \begin{pmatrix}
        0\\
        1
      \end{pmatrix}.
    \end{align*}}}
  \end{enumerate}

  \item \textbf{谐振子问题}

  \textbf{一维谐振子的哈密顿量为
  \begin{align*}
    H = \frac{p^{2}}{2m} + \frac{1}{2}m\omega^{2}x^{2}
  \end{align*}
  坐标算符 $x$ 和动量算符 $p$ 满足对易式 $[x,p] = i\hbar$. 对动量算符和坐标算符进行重新标度
  \begin{align*}
    p = P\sqrt{\hbar m\omega},\quad x = Q\sqrt{\frac{\hbar}{m\omega}}
  \end{align*}
  注意新的坐标算符 $Q$ 和动量算符 $P$ 是无量纲的, 哈密顿量重新写为
  \begin{align*}
    H = \frac{1}{2}\hbar\omega(P^{2} + Q^{2})
  \end{align*}
  引入玻色子产生和湮灭算符, $a^{\dagger}$ 和 $a$.
  \begin{align*}
    a = \frac{1}{\sqrt{2}}\left(Q + iP\right),\quad a^{\dagger} = \frac{1}{\sqrt{2}}\left(Q - iP\right)
  \end{align*}}
  \begin{enumerate}
    \item \textbf{计算 $[Q,P]$, $[a,a^{\dagger}]$, $[a,a^{\dagger}a]$, $[a^{\dagger},a^{\dagger}a]$;}
    
{\color{gray}{    \begin{align*}
      [Q,P] &= [\sqrt{\frac{m\omega}{\hbar}}x,\sqrt{\frac{1}{\hbar m\omega}}p] = \frac{1}{\hbar}[x,p] = \frac{1}{\hbar} i\hbar = \boxed{i},\\
      [a,a^{\dagger}] &= \left[\frac{1}{\sqrt{2}}(Q + iP),\frac{1}{\sqrt{2}}(Q - iP)\right] \\
      &= \frac{1}{2}[Q + iP, Q - iP]= \frac{1}{2}\left([Q,Q] - i[Q, P] + i[P,Q] + [P,P]\right)\\
      &= \frac{1}{2}[0 - i\cdot i + i\cdot (-i) + 0] = \boxed{1},\\
      [a,a] &= \left[\frac{1}{\sqrt{2}}(Q + iP),\frac{1}{\sqrt{2}}(Q + iP)\right]\\
      &= \frac{1}{2}[Q + iP, Q + iP] = \frac{1}{2}\left([Q,Q] + i[Q,P] + i[P,Q] + [P,P]\right)\\
      &= \frac{1}{2}[0 + i\cdot i + i\cdot (-i) + 0] = 0,\\
      [a^{\dagger},a^{\dagger}] &= \left[\frac{1}{\sqrt{2}}(Q - iP),\frac{1}{\sqrt{2}}(Q - iP)\right]\\
      &= \frac{1}{2}[Q - iP, Q-iP] = \frac{1}{2}([Q,Q]-i[Q,P]-i[P,Q] + [P,P])\\
      &= \frac{1}{2}(0 - i\cdot i - i\cdot(-i) + 0) = 0,\\
      [a,a^{\dagger}a] &= a^{\dagger}[a,a] + [a,a^{\dagger}]a = a^{\dagger}\cdot 0 + 1\cdot a = \boxed{a},\\
      [a^{\dagger},a^{\dagger}a] &= a^{\dagger}[a^{\dagger},a] + [a^{\dagger},a^{\dagger}]a = a^{\dagger}\cdot (-1) + 0\cdot a = \boxed{-a^{\dagger}}.
    \end{align*}}}
    \item \textbf{将哈密顿量 $H$ 用 $a$ 和 $a^{\dagger}$ 表示. 并求出全部能级;}
    
{\color{gray}{    \begin{align*}
      a &= \frac{1}{\sqrt{2}}\left(Q + iP\right),\quad a^{\dagger} = \frac{1}{\sqrt{2}}\left(Q - iP\right)\\
      \Rightarrow Q &= \frac{1}{\sqrt{2}}(a + a^{\dagger}),\quad P = \frac{1}{\sqrt{2}i}(a - a^{\dagger})\\
      \Rightarrow H &= \frac{1}{2}\hbar\omega(P^{2} + Q^{2}) = \frac{1}{2}\hbar\omega\left\{\left[\frac{1}{\sqrt{2}i}(a - a^{\dagger})\right]^{2} + \left[\frac{1}{\sqrt{2}}(a + a^{\dagger})\right]^{2}\right\}\\
      &= \frac{1}{2}\hbar\omega\left\{
        -\frac{1}{2}\left(aa - aa^{\dagger} - a^{\dagger}a + a^{\dagger}a^{\dagger}\right) + 
        \frac{1}{2}\left(aa + aa^{\dagger} + a^{\dagger}a + a^{\dagger}a^{\dagger}\right)
      \right\}\\
      &= \frac{1}{2}\hbar\omega\left(a^{\dagger}a + aa^{\dagger}\right)
    \end{align*}
    当然, 也可以利用 $\begin{aligned}
      [a,a^{\dagger}] =1\iff aa^{\dagger} = a^{\dagger}a + 1
    \end{aligned}$ 将 $H$ 变换为熟知的粒子数表象形式:
    \begin{align*}
      \boxed{H = \hbar\omega\left(a^{\dagger}a + \frac{1}{2}\right)}
    \end{align*}
    所以 $\begin{aligned}
      \boxed{E_{n} = \hbar\omega\left(n + \frac{1}{2}\right),\quad n = 0,1,2,\cdots}
    \end{aligned}$}}

    \item \textbf{在能量表象中, 计算 $a$ 和 $a^{\dagger}$ 的矩阵元.}
    
{\color{gray}{    能量表象的本征矢满足 $H|n\rangle = E_{n}|n\rangle$, 则矩阵元为
    \begin{align*}
      a|n\rangle &= \sqrt{n}|n-1\rangle,\quad a^{\dagger}|n\rangle = \sqrt{n+1}|n+1\rangle\\
      \Rightarrow \langle m|a|n\rangle &= \boxed{\sqrt{n}\delta_{m,n-1}},\quad \langle m|a^{\dagger}|n\rangle = \boxed{\sqrt{n+1}\delta_{m,n+1}}
    \end{align*}}}
  \end{enumerate}

  \item \textbf{角动量耦合}
  
\textbf{两个大小相等, 属于不同自由度的角动量 $\vec{J}_{1}$ 和 $\vec{J}_{2}$ 耦合成总角动量 $\vec{J} = \vec{J}_{1} + \vec{J}_{2}$, 设 $\vec{J}_{1}^{2} = \vec{J}_{2}^{2} = j(j+1)\hbar^{2}$, $J^{2} = J(J+1)\hbar^{2}$, $J = 2j, 2j-1, \cdots, 1, 0$. 在总角动量量子数 $J = 0$ 的状态下, 求 $J_{1,z}$ 和 $J_{2,z}$ 的可能取值及相应概率.}

{\color{gray}{根据 $J = 0$, 而 $-|J|\leq M\leq |J|$, 夹逼定理得到 $M= 0$. 而磁量子数守恒, 所以 $J_{1,z} + J_{2,z} = J_{z} = 0$. 已知 C-G 系数可以用于将 $|J,M;j_{1},j_{2}\rangle$ 以基矢 $|j_{1},m_{1};j_{2},m_{2}\rangle$ 展开, 代入上述讨论结果有

\begin{align*}
  |0,0;j,j\rangle = \sum_{m_,-m}^{-j\leq m\leq j}C_{j,j,m,-m}^{0,0}|j,m;j,-m\rangle
\end{align*}

概率即为 $P(m_{1}=m,m_{2}=-m) = |C_{j,j,m,-m}^{0,0}|^{2}$. 那么问题就来到如何计算这个特殊的 C-G 系数. 根据 C-G 系数的递推定义, 可以得到其解析表达式

\begin{align*}
  &\langle j_{1},m_{1};j_{2},m_{2}|J,M;j_{1},j_{2}\rangle \\
  &=\sqrt{\frac{(2J+1)(J+j_{1}-j_{2})!(J-j_{1}+j_{2})!(j_{1}+j_{2}-J)!}{(j_{1}+j_{2}+J+1)!}}\\
  &\times \sqrt{(J+M)!(J-M)!(j_{1}+m_{1})!(j_{1}-m_{1})!(j_{2}+m_{2})!(j_{2}-m_{2})!}\\
  &\times \sum_{k_{\text{min}}}^{k_{\text{max}}}\frac{(-1)^{k}}{k!(j_{1}+j_{2}-J-k)!(j_{1}-m_{1}-k)!(j_{2}+m_{2}-k)!(J-M-k)!}\\
  &\times \frac{1}{(J-j_{2}+m_{1}+k)!(J-j_{1}-m_{2}+k)!}\\
  k_{\text{min}} &= \text{max}\{0,j_{2}-m_{1}-J,j_{1}+m_{2}-J\},\quad
  k_{\text{max}} = \text{min}\{j_{1}+j_{2}-J, j_{1}-m_{1},j_{2}+m_{2}\}
\end{align*}

所以代入 $j_{1}=j_{2}=j$, $m_{1}=-m_{2}=m$, 即有 $\begin{aligned}
  C_{j,m,j,-m}^{0,0} = \frac{(-1)^{j-m}}{\sqrt{2j+1}}
\end{aligned}$, 显然因为平方消去了可能存在的负号, 使得 
$|j,m;j,-m\rangle,\quad\forall m\in \{-j,-j+1,\cdots,j-1,j\}$ 等概率, 所以得到

\begin{align*}
  \boxed{P(m_{1}=m,m_{2}=-m) = \frac{1}{2j+1}}
\end{align*}}}

  \item \textbf{自旋-1 模型}
  
\textbf{考虑自旋-1 体系, 自旋算符为 $\vec{S}$, 考虑 $(\vec{S}^{2},S^{z})$ 表象, 基矢顺序为 $|1,1\rangle$, $|1,0\rangle$, $|1,-1\rangle$, 简记为 $|+1\rangle$, $|0\rangle$, $|-1\rangle$. 设 $\hbar = 1$. }
  \begin{enumerate}
    \item \textbf{写出 $S^{x}$ 和 $S^{z}$ 的矩阵表示.}
    
{\color{gray}{由于是在 $(\vec{S}^{2},S^{z})$ 表象, 所以 $S^{z}$ 的矩阵一定是对角矩阵. 选定基矢为 $\{|s,m\rangle\}$, 即$|1,1\rangle=\begin{pmatrix}
      1\\
      0\\
      0
    \end{pmatrix}$, $|1,0\rangle = \begin{pmatrix}
      0\\
      1\\
      0
    \end{pmatrix}$, $|1,-1\rangle = \begin{pmatrix}
      0\\
      0\\
      1
    \end{pmatrix}$. 根据本征方程 $S^{z}|s,m\rangle = m|s,m\rangle$, 得到
    \begin{align*}
      S^{z} = \boxed{\begin{pmatrix}
        1 & 0 & 0\\
        0 & 0 & 0\\
        0 & 0 & -1
      \end{pmatrix}}
    \end{align*} 
    而对于 $S^{x}$ (包括题解不要求的 $S^{y}$), 我们实际上是使用的升降算符 $S^{\pm}$ 来定义的. 
    \begin{align*}
      S^{+}|s,m\rangle &= \sqrt{s(s+1)-m(m+1)}|s,m+1\rangle,\\
      S^{-}|s,m\rangle &= \sqrt{s(s+1)-m(m-1)}|s,m-1\rangle.\\
      \Rightarrow S^{+}|1,1\rangle &= 0,\quad S^{+}|1,0\rangle = \sqrt{2} |1,1\rangle,\quad S^{+}|1,-1\rangle = \sqrt{2}|1,0\rangle,\\
      S^{-}|1,1\rangle &= \sqrt{2}|1,0\rangle, \quad S^{-}|1,0\rangle = \sqrt{2}|1,-1\rangle, \quad S^{-}|1,-1\rangle = 0.\\
      \Rightarrow S^{+} &= \begin{pmatrix}
        0 & \sqrt{2} & 0\\
        0 & 0 & \sqrt{2}\\
        0 & 0 & 0
      \end{pmatrix},\quad S^{-} = \begin{pmatrix}
        0 & 0 & 0\\
        \sqrt{2} & 0 & 0\\
        0 & \sqrt{2} & 0
      \end{pmatrix}.\\
      \Rightarrow S^{x} &= \frac{1}{2}\left(S^{+} + S^{-}\right) = \boxed{\frac{1}{\sqrt{2}}\begin{pmatrix}
        0 & 1 & 0\\
        1 & 0 & 1\\
        0 & 1 & 0
      \end{pmatrix}}
    \end{align*}}}
    \item \textbf{考虑哈密顿量 $H(\lambda) = H_{0} + \lambda V$, 其中 $H_{0} = (S^{z})^{2}$, $V = S^{x} + S^{z}$. 考虑为 $\lambda V$ 微扰, 利用微扰论计算微扰后的各能级和各能态, 其中能级微扰准确到二阶, 能态微扰准确到一阶.}
{\color{gray}{
  
% \begin{align*}
%       H_{0}|s,m\rangle &= \left(S^{z}\right)^{2}|s,m\rangle = m^{2}|s,m\rangle\\
%       \Rightarrow E_{-1}^{(0)} & = 1,\quad E_{0} = 0,\quad E_{1} = 1
%     \end{align*}
%     注意到 $m^{2}$ 会带来 $m=\pm 1$ 的简并, 所以后续计算时会涉及简并态的微扰处理. 首先观察简并态, 简并态矢张成独立子空间, 于是求解这个子空间中 $V$ 的矩阵:
%     \begin{align*}
%       V_{\text{sub}} &= \begin{pmatrix}
%         \langle 1,1|V|1,1\rangle & \langle 1,1|V|1,-1\rangle\\
%         \langle 1,-1|V|1,1\rangle & \langle 1,-1|V|1,-1\rangle
%       \end{pmatrix}\\
%       \langle 1,1|V|1,1\rangle &= \begin{pmatrix}
%         1 & 0 & 0
%       \end{pmatrix}\begin{pmatrix}
%         1 & \frac{1}{\sqrt{2}}& \\
%         \frac{1}{\sqrt{2}} & 0  & \frac{1}{\sqrt{2}}\\
%           & \frac{1}{\sqrt{2}} & -1
%        \end{pmatrix}\begin{pmatrix}
%           1\\
%           0\\
%           0
%        \end{pmatrix} = 1,\\
%        \langle 1,1|V|1,-1\rangle &= \begin{pmatrix}
%         1 & 0 & 0
%       \end{pmatrix}\begin{pmatrix}
%         1 & \frac{1}{\sqrt{2}}& \\
%         \frac{1}{\sqrt{2}} & 0  & \frac{1}{\sqrt{2}}\\
%           & \frac{1}{\sqrt{2}} & -1
%        \end{pmatrix}\begin{pmatrix}
%           0\\
%           0\\
%           1
%         \end{pmatrix} = 0,\\
%         \langle 1,-1|V|1,1\rangle &= 0,\\
%         \langle 1,-1|V|1,-1\rangle &= \begin{pmatrix}
%           0 & 0 & 1
%         \end{pmatrix}\begin{pmatrix}
%           1 & \frac{1}{\sqrt{2}}& \\
%           \frac{1}{\sqrt{2}} & 0  & \frac{1}{\sqrt{2}}\\
%             & \frac{1}{\sqrt{2}} & -1
%          \end{pmatrix}\begin{pmatrix}
%             0\\
%             0\\
%             1
%          \end{pmatrix} = -1.\\
%          \Rightarrow V_{\text{sub}} &= \begin{pmatrix}
%           1 & 0\\
%           0 & -1
%          \end{pmatrix}
%     \end{align*}
%     注意到计算得到的子空间中 $V_{\text{sub}}$ 完成了对角化, 这说明沿用的 $|s,m\rangle$ 基矢已经是 "好量子态". 所以回归到非简并微扰论的方法. 一阶能量修正各为
%     \begin{align*}
%       E_{1}^{(1)} &= \langle 1,1|V|1,1\rangle = \begin{pmatrix}
%         1 & 0 & 0
%       \end{pmatrix}\begin{pmatrix}
%         1 & \frac{1}{\sqrt{2}}& \\
%         \frac{1}{\sqrt{2}} & 0  & \frac{1}{\sqrt{2}}\\
%           & \frac{1}{\sqrt{2}} & -1
%        \end{pmatrix}\begin{pmatrix}
%           1\\
%           0\\
%           0
%        \end{pmatrix} = \boxed{1},\\
%        E_{0}^{(1)} &= \langle 1,0|V|1,0\rangle = \begin{pmatrix}
%         0 & 1 & 0
%       \end{pmatrix}\begin{pmatrix}
%         1 & \frac{1}{\sqrt{2}}& \\
%         \frac{1}{\sqrt{2}} & 0  & \frac{1}{\sqrt{2}}\\
%           & \frac{1}{\sqrt{2}} & -1
%        \end{pmatrix}\begin{pmatrix}
%           0\\
%           1\\
%           0
%        \end{pmatrix} = \boxed{0},\\
%        E_{-1}^{(1)} &= \langle 1,-1|V|1,-1\rangle = \begin{pmatrix}
%         0 & 0 & 1
%       \end{pmatrix}\begin{pmatrix}
%         1 & \frac{1}{\sqrt{2}}& \\
%         \frac{1}{\sqrt{2}} & 0  & \frac{1}{\sqrt{2}}\\
%           & \frac{1}{\sqrt{2}} & -1
%        \end{pmatrix}\begin{pmatrix}
%           0\\
%           0\\
%           1
%        \end{pmatrix} = \boxed{-1},\\
%     \end{align*}
%     二阶能量修正由公式 $\begin{aligned}
%       E_{m}^{(n)} = \sum_{n\neq m}\frac{|\langle n|V|m\rangle|^{2}}{E_{m}^{(0)} - E_{n}^{(0)}}
%     \end{aligned}$ 给出:
%     \begin{align*}
%       E_{1}^{(2)} &= \frac{|\langle 1,0|V|1,1\rangle|^{2}}{E_{1}^{(0)}-E_{0}^{0}} + \frac{|\langle 1,-1|V|1,1\rangle|^{2}}{E_{1}^{(0)} - E_{-1}^{(0)}} = \frac{\left(\frac{1}{\sqrt{2}}\right)^{2}}{1 - 0} + \frac{0^{2}}{1 - 1} = \boxed{\frac{1}{2}},\\
%       E_{0}^{(2)} &= \frac{|\langle 1,1|V|1,0\rangle|^{2}}{E_{0}^{(0)} - E_{1}^{(0)}} + \frac{|\langle 1,-1|V|1,0\rangle|^{2}}{E_{0}^{(0)} - E_{-1}^{(0)}} = \frac{\left(\frac{1}{\sqrt{2}}\right)^{2}}{0-1} + \frac{0^{2}}{0 - 1} = \boxed{-\frac{1}{2}},\\
%       E_{-1}^{(2)} &= \frac{|\langle 1,0|V|1,-1\rangle|^{2}}{E_{-1}^{(0)} - E_{0}^{(0)}} + \frac{|\langle 1,1|V|1,-1\rangle|^{2}}{E_{-1}^{(0)} - E_{1}^{(0)}} = \frac{\left(\frac{1}{\sqrt{2}}\right)^{2}}{1-0} + \frac{0^{2}}{1 - 1} = \boxed{\frac{1}{2}}
%     \end{align*}
%     可见, 只要在 $E_{i}^{(1)} - E_{j}^{(1)}=0$ 时分子也为 $0$, 我们就可以无视分母为 $0$ 的问题. 
%     接下来是对态函数的微扰修正. 一阶修正由 $\begin{aligned}
%       |m\rangle^{(1)} = \sum_{n\neq m}|n\rangle\frac{\langle n|V|m\rangle}{E_{m}^{(0)} - E_{n}^{(0)}}
%     \end{aligned}$ 给出:
%     \begin{align*}
%       |1,1\rangle^{(1)} &= |1,0\rangle\frac{\langle 1,0|V|1,1\rangle}{E_{1}^{(0)} - E_{0}^{(0)}} + |1,-1\rangle\frac{\langle 1,-1|V|1,1\rangle}{E_{1}^{(0)} - E_{-1}^{(0)}} = |1,0\rangle\frac{1}{\sqrt{2}}\frac{1}{1 - 0} + |1,-1\rangle\cdot 0 \\
%       &= \boxed{\frac{1}{\sqrt{2}}|1,0\rangle}\\
%       |1,0\rangle^{(1)} &= |1,1\rangle\frac{\langle 1,1|V|1,0\rangle}{E_{0}^{(0)} - E_{1}^{(0)}} + |1,-1\rangle\frac{\langle 1,-1|V|1,0\rangle}{E_{0}^{(0)} - E_{-1}^{(0)}} = |1,1\rangle \frac{1}{\sqrt{2}}\frac{1}{0-1} + |1,-1\rangle \frac{1}{\sqrt{2}}\cdot\frac{1}{0-1} \\
%       &= \boxed{-\frac{1}{\sqrt{2}}(|1,1\rangle + |1,-1\rangle)}\\
%       |1,-1\rangle^{(1)} &= |1,1\rangle\frac{\langle 1,1|V|1,-1\rangle}{E_{-1}^{(0)}-E_{1}^{(0)}} + |1,0\rangle\frac{\langle 1,0|V|1,-1\rangle}{E_{-1}^{(0)}-E_{0}^{(0)}} = |1,1\rangle\cdot 0 + |1,0\rangle\frac{1}{\sqrt{2}}\cdot\frac{1}{1-0} \\
%       &= \boxed{\frac{1}{\sqrt{2}}|1,0\rangle}
%     \end{align*}
%     总结:
%     \begin{align*}
%       E_{1} & = 1 + 1\lambda + \frac{1}{2}\lambda^{2} + \mathit{o}(\lambda^{2})\\
%       E_{0} & = 0 + 0\lambda - \frac{1}{2}\lambda^{2} +  \mathit{o}(\lambda^{2})\\
%       E_{-1} &= 1 - 1\lambda + \frac{1}{2}\lambda^{2} + \mathit{o}(\lambda^{2})\\
%       |1,1\rangle &= |1,1\rangle + \frac{\lambda}{\sqrt{2}}|1,0\rangle + \mathit{o}(\lambda)\\
%       |1,0\rangle &= |1,0\rangle - \frac{\lambda}{\sqrt{2}}(|1,1\rangle + |1,-1\rangle) + \mathit{o}(\lambda)\\
%       |1,-1\rangle &= |1,-1\rangle + \frac{\lambda}{\sqrt{2}}|1,0\rangle + \mathit{o}(\lambda)
%     \end{align*}
%     对于这类可以使用矩阵形式讨论的问题, 还有一种笨办法, 就是直接严格对角化含 $\lambda$ 微扰的哈密顿量, 然后进行 Taylor 展开得到各级数. 但是在三阶矩阵下的计算已经非常复杂, 所以还是建议使用一般微扰论方法, 毕竟考试时是会给出公式的.
    
    首先计算 $H_{0}$ 的本征矢和本征值:
    \begin{align*}
        \begin{matrix}
            n & \text{states} & E_{n}\\
            1 & |\stackrel{n}{1}, \stackrel{\alpha}{+1}\rangle = |\psi_{1}\rangle,\quad |\stackrel{n}{1},\stackrel{\alpha}{-1}\rangle = |\psi_{3}\rangle & E_{1} = 1\\
            0 & |\stackrel{n}{0},\stackrel{\alpha}{0}\rangle = |\psi_{2}\rangle & E_{0} = 0\\
        \end{matrix}
    \end{align*}
    计算波函数的修正:
    \begin{align*}
        |\stackrel{n}{1},\stackrel{\alpha}{\pm 1}\rangle^{\prime} &= |\stackrel{n}{1},\stackrel{\alpha}{\pm 1}\rangle 
        + |\stackrel{m}{0},\stackrel{\beta}{0}\rangle\frac{V_{\stackrel{m}{0}\stackrel{\beta}{0},\stackrel{n}{0}\stackrel{\alpha}{\pm 1}}}{E^{(\stackrel{n}{1})} - E^{(\stackrel{m}{0})}}\lambda + \cdots\\
        &= |\stackrel{n}{1},\stackrel{\alpha}{\pm 1}\rangle + |\stackrel{m}{0},\stackrel{\beta}{0}\rangle\frac{1}{\sqrt{2}}\lambda + \cdots\\
        |\stackrel{n}{0},\stackrel{\alpha}{0}\rangle^{\prime} &= |\stackrel{n}{0},\stackrel{\alpha}{0}\rangle
        + |\stackrel{m}{1},\stackrel{\beta}{+1}\rangle\frac{V_{\stackrel{m}{1}\stackrel{\beta}{+1}, \stackrel{n}{0}\stackrel{\alpha}{0}}}{E^{\stackrel{n}{(0)}} - E^{\stackrel{m}{(1)}}}\lambda
        + |\stackrel{m}{1},\stackrel{\beta}{-1}\rangle\frac{V_{\stackrel{m}{1}\stackrel{\beta}{-1}, \stackrel{n}{0}\stackrel{\alpha}{0}}}{E^{\stackrel{n}{(0)}} - E^{\stackrel{m}{(1)}}}\lambda + \cdots\\
        &= |\stackrel{n}{0},\stackrel{\alpha}{0}\rangle
         - (|\stackrel{m}{1},\stackrel{\beta}{+1}\rangle + |\stackrel{m}{1},\stackrel{\beta}{-1}\rangle)\frac{1}{\sqrt{2}}\lambda + \cdots
    \end{align*}
    可见在 $n=1$ 存在简并子空间. 选定 $|\stackrel{n}{1},\stackrel{\alpha}{+1}\rangle^{\prime}$ 和 $|\stackrel{n}{1},\stackrel{\alpha}{-1}\rangle$ 作为基矢. 有效哈密顿量的矩阵元为
    \begin{align*}
        E^{(\stackrel{n}{1})}_{\stackrel{\alpha}{+1},\stackrel{\beta}{+1}} &= 
        E^{(\stackrel{n}{1})} 
        + V_{\stackrel{n}{1}\stackrel{\alpha}{+1},\stackrel{n}{1}\stackrel{\beta}{+1}}\lambda 
        + \frac{V_{\stackrel{n}{1}\stackrel{\alpha}{+1},\stackrel{m}{0}\stackrel{\gamma}{0}}V_{\stackrel{m}{0}\stackrel{\gamma}{0},\stackrel{n}{1}\stackrel{\beta}{+1}}}{E^{(\stackrel{n}{1})} - E^{(\stackrel{m}{0})}}\lambda^{2} = 1 + \lambda + \frac{\lambda^{2}}{2}\\
        E^{(\stackrel{n}{1})}_{\stackrel{\alpha}{-1},\stackrel{\beta}{-1}} &= 
        E^{(\stackrel{n}{1})}
        + V_{\stackrel{n}{1}\stackrel{\alpha}{-1},\stackrel{n}{1}\stackrel{\beta}{-1}}\lambda
        + \frac{V_{\stackrel{n}{1}\stackrel{\alpha}{-1},\stackrel{m}{0}\stackrel{\gamma}{0}}V_{\stackrel{m}{0}\stackrel{\gamma}{0},\stackrel{n}{1}\stackrel{\beta}{-1}}}{E^{(\stackrel{n}{1})} - E^{(\stackrel{m}{0})}}\lambda^{2} = 1 - \lambda + \frac{\lambda^{2}}{2}\\
        E^{(\stackrel{n}{1})}_{\stackrel{\alpha}{+1},\stackrel{\beta}{-1}} &=
        V_{\stackrel{n}{1}\stackrel{\alpha}{+1},\stackrel{n}{1}\stackrel{\beta}{-1}}\lambda + \frac{V_{\stackrel{n}{1}\stackrel{\alpha}{+1},\stackrel{m}{0}\stackrel{\gamma}{0}}V_{\stackrel{m}{0}\stackrel{\gamma}{0},\stackrel{n}{1}\stackrel{\beta}{-1}}}{E^{(\stackrel{n}{1})} - E^{(\stackrel{m}{0})}}\lambda^{2} =  \frac{\lambda^{2}}{2}\\
        E^{(\stackrel{n}{1})}_{\stackrel{\alpha}{-1},\stackrel{\beta}{+1}} &=
        V_{\stackrel{n}{1}\stackrel{\alpha}{-1},\stackrel{n}{1}\stackrel{\beta}{+1}}\lambda + \frac{V_{\stackrel{n}{1}\stackrel{\alpha}{-1},\stackrel{m}{0}\stackrel{\gamma}{0}}V_{\stackrel{m}{0}\stackrel{\gamma}{0},\stackrel{n}{1}\stackrel{\beta}{+1}}}{E^{(\stackrel{n}{1})} - E^{(\stackrel{m}{0})}}\lambda^{2} =  \frac{\lambda^{2}}{2}
    \end{align*}
    有效哈密顿量为 $\begin{aligned}
        H_{\stackrel{n}{1}}^{\text{eff}} = \begin{pmatrix}
            1 + \lambda + \frac{\lambda^{2}}{2} & \frac{\lambda^{2}}{2}\\
            \frac{\lambda^{2}}{2} & 1 - \lambda + \frac{\lambda^{2}}{2}
        \end{pmatrix}
    \end{aligned}$, 此时对角元已经不等, 说明简并已经解除. 那么在这个更小的子空间中, 进一步使用微扰, 即一阶修正后的能量和波函数视为原始哈密顿量和波函数:
    \begin{align*}
        H_{\stackrel{n}{1}}^{\text{eff}} &= \begin{pmatrix}
            1 + \lambda + \frac{\lambda^{2}}{2} & \frac{\lambda^{2}}{2}\\
            \frac{\lambda^{2}}{2} & 1 - \lambda + \frac{\lambda^{2}}{2}
        \end{pmatrix} = \underbrace{\begin{pmatrix}
            1 + \lambda + \frac{\lambda^{2}}{2}  & 0\\
            0 & 1 - \lambda + \frac{\lambda^{2}}{2}
        \end{pmatrix}}_{H_{0}^{\prime}} + \underbrace{\begin{pmatrix}
            0 & \frac{\lambda^{2}}{2}\\
            \frac{\lambda^{2}}{2} & 0
        \end{pmatrix}}_{V^{\prime}}\\
        |\stackrel{n}{1},\stackrel{\alpha}{+1}\rangle^{\prime\prime} &= |\stackrel{n}{1},\stackrel{\alpha}{+1}\rangle^{\prime} + |\stackrel{n}{1},\stackrel{\beta}{-1}\rangle^{\prime}\frac{V^{\prime}_{\stackrel{\beta}{-1},\stackrel{\alpha}{+1}}}{E^{\prime}_{\stackrel{\alpha}{+1}} - E^{\prime}_{\stackrel{\beta}{-1}}} + \cdots
        = |\stackrel{n}{1},\stackrel{\alpha}{+1}\rangle^{\prime} + |\stackrel{n}{1},\stackrel{\beta}{-1}\rangle^{\prime}\frac{\lambda}{4} + \cdots\\
        |\stackrel{n}{1},\stackrel{\alpha}{-1}\rangle^{\prime\prime} &= |\stackrel{n}{1},\stackrel{\alpha}{-1}\rangle^{\prime} + |\stackrel{n}{1},\stackrel{\beta}{+1}\rangle^{\prime}\frac{V^{\prime}_{\stackrel{\beta}{+1},\stackrel{\alpha}{-1}}}{E^{\prime}_{\stackrel{\alpha}{-1}} - E^{\prime}_{\stackrel{\beta}{+1}}} + \cdots
        = |\stackrel{n}{1},\stackrel{\alpha}{-1}\rangle^{\prime} - |\stackrel{n}{1},\stackrel{\beta}{+1}\rangle^{\prime}\frac{\lambda}{4} + \cdots
    \end{align*}
    代入 $|\stackrel{n}{1},\stackrel{\alpha}{\pm 1}\rangle^{\prime}$ 即可得到进一步考虑了简并微扰的波函数, 注意要忽略 $\lambda^{2}$ 阶:
    \begin{align*}
        |\stackrel{n}{1},\stackrel{\alpha}{+1}\rangle^{\prime\prime} &= |\stackrel{n}{1},\stackrel{\alpha}{+1}\rangle + |\stackrel{n}{1},\stackrel{\beta}{0}\rangle\frac{\lambda}{\sqrt{2}} + |\stackrel{n}{1},\stackrel{\beta}{-1}\rangle\frac{\lambda}{4}\\
        |\stackrel{n}{1},\stackrel{\alpha}{-1}\rangle^{\prime\prime} &= |\stackrel{n}{1},\stackrel{\alpha}{-1}\rangle + |\stackrel{n}{1},\stackrel{\beta}{0}\rangle\frac{\lambda}{\sqrt{2}} - |\stackrel{n}{1},\stackrel{\beta}{+1}\rangle\frac{\lambda}{4}
    \end{align*}
    能量修正:
    \begin{align*}
        E^{\prime\prime}_{\stackrel{n}{1},\stackrel{\alpha}{+1}} &= E^{\prime}_{\stackrel{n}{1},\stackrel{\alpha}{+1}} + V^{\prime}_{\stackrel{n}{1},\stackrel{\alpha}{+1}\stackrel{\alpha}{+1}} + \frac{V^{\prime}_{\stackrel{n}{1},\stackrel{\alpha}{+1}\stackrel{\beta}{-1}}V^{\prime}_{\stackrel{n}{1},\stackrel{\beta}{-1}\stackrel{\alpha}{+1}}}{E^{\prime}_{\stackrel{n}{1},\stackrel{\alpha}{+1}} - E^{\prime}_{\stackrel{n}{1},\stackrel{\beta}{-1}}} = 1 + \lambda + \frac{\lambda^{2}}{2} + \mathcal{O}(\lambda^{3})\\
        E^{\prime\prime}_{\stackrel{n}{1},\stackrel{\alpha}{-1}} &= E^{\prime}_{\stackrel{n}{1},\stackrel{\alpha}{-1}} + V^{\prime}_{\stackrel{n}{1},\stackrel{\alpha}{-1}\stackrel{\alpha}{-1}} + \frac{V^{\prime}_{\stackrel{n}{1},\stackrel{\alpha}{-1}\stackrel{\beta}{+1}}V^{\prime}_{\stackrel{n}{1},\stackrel{\beta}{+1}\stackrel{\alpha}{-1}}}{E^{\prime}_{\stackrel{n}{1},\stackrel{\alpha}{-1}} - E^{\prime}_{\stackrel{n}{1},\stackrel{\beta}{+1}}} = 1 - \lambda + \frac{\lambda^{2}}{2} + \mathcal{O}(\lambda^{3})
    \end{align*}
    }}
  \end{enumerate}

  \item \textbf{均匀电子气}

\textbf{考虑三维相互作用均匀电子气, 哈密顿量为 $H = H_{0} + H_{I}$. 考虑系统体积为 $V = L^{3}$, 每个方向的系统尺寸为 $L$. 采用箱归一化, 所以 $\vec{k}$ 是离散的, $\begin{aligned}\vec{k} = \frac{2\pi}{L}(n_{x},n_{y},n_{z})\end{aligned}$, $n_{x}$, $n_{y}$, $n_{z}$ 为整数. 采用二次量子化的语言, 可给出哈密顿量在动量空间的形式. $H_{0}$ 为单体部分:}
  \begin{align*}
    H_{0} = \sum_{\vec{k}\sigma}\varepsilon_{\vec{k}}c_{\vec{k}\sigma}^{\dagger}c_{\vec{k}\sigma}
  \end{align*}
\textbf{其中 $\begin{aligned}\varepsilon_{\vec{k}} = \frac{\hbar^{2}\vec{k}^{2}}{2m}\end{aligned}$ 是自由电子的色散关系. 用 $\varepsilon_{F}$ 表示费米能, $k_{F}$ 表示费米波矢的大小.}

  \textbf{$H_{I}$ 为两体相互作用部分,
  \begin{align*}
    H_{I} = \frac{1}{2V}\sum_{\vec{k}_{1},\vec{k}_{2},\vec{q}}\sum_{\sigma\sigma^{\prime}}v(q)c_{\vec{k}_{1}+\vec{q},\sigma}^{\dagger}c_{\vec{k}_{2}-\vec{q},\sigma^{\prime}}^{\dagger}c_{\vec{k}_{2}\sigma^{\prime}}c_{\vec{k}_{1}\sigma}
  \end{align*}
  $v(q)$ 是相互作用 $v(x)$ 的傅里叶变换形式, $q = |\vec{q}|$, $x = |\vec{x}|$,
  \begin{align*}
    v(q) = \frac{1}{V}\int v(x)e^{-i\vec{q}\cdot\vec{x}}\mathrm{d}^{3}\vec{x}
  \end{align*}
  这里我们考虑短程势, 也就是说 $v(q=0)$ 不发散.}

  \textbf{自由电子气零温下处于电子填充到费米能 $\varepsilon_{F}$ 的费米海态(Fermi sea state), 简记为 FS, 利用费米子产生算符作用到真空态上可以表示 FS 态为
  \begin{align*}
    |\text{FS}\rangle = \prod_{k < k_{F},\sigma}c_{\vec{k}\sigma}^{\dagger}|0\rangle
  \end{align*}}
  \begin{enumerate}
    \item \textbf{考虑零温下的自由电子气, 计算总粒子数 $N$ 和粒子数密度 $n$, 计算总能量 $E^{(0)}$ 并把总能量密度 $E^{(0)}/V$ 表示成粒子数密度 $n$ 的函数. }\label{final-2022-5}
    
{\color{gray}{分离变量法求解薛定谔方程 $\begin{aligned}
      \frac{\hbar^{2}\hat{k}^{2}}{2m}\psi = E\psi
    \end{aligned}$. 于是能量本征值为 $\begin{aligned}
      \frac{\hbar^{2}k^{2}}{2m} = \sum_{i}\frac{\hbar^{2}k_{i}^{2}}{2m}
    \end{aligned}$, 其中 $\begin{aligned}
      k_{i} = \frac{\sqrt{2mE_{i}}}{\hbar}
    \end{aligned}$. 由于使用了箱归一化, 即有边界条件 $\begin{aligned}
      k_{i}l_{i} = n_{i}\pi(n_{i}\in\mathbb{N}^{*})
    \end{aligned}$, 代入即得
    \begin{align*}
      E = \frac{\hbar^{2}}{2m}\left[\sum_{i}^{3}\left(\frac{\pi}{l_{i}}\right)^{2}n_{i}^{2}\right] = \frac{\hbar^{2}\pi^{2}}{2m}\left(\sum_{i}^{3}\frac{n_{i}^{2}}{l_{i}^{2}}\right)
    \end{align*}
    每个波矢 $\begin{aligned}
      \vec{k} = \left(\frac{\pi}{l_{x}}n_{x},\frac{\pi}{l_{y}}n_{y},\frac{\pi}{l_{z}}n_{z}\right)
    \end{aligned}$ 都是在 $\vec{k}$ 空间中的一个格点, 这种格点所占据的 $\vec{k}$ 空间体积为 
    
    $\begin{aligned}
      \prod_{i}^{3}\frac{\pi}{l_{i}} = \frac{\pi^{3}}{l_{x}l_{y}l_{z}} = \frac{\pi^{3}}{V}
    \end{aligned}$, 其中 $V$ 代表了物质在 $\vec{x}$ 空间的体积(实体积). 电子是全同费米子, 每个格点上(每个状态)能且只能容纳两个电子. 而费米-狄拉克分布为$\begin{aligned}
      f(\epsilon) = \frac{1}{1 + e^{\beta(\epsilon - \mu)}}
    \end{aligned}$. 绝对零度($\beta\rightarrow\infty$)下, 电子可占据的最高能级即为费米能级 $\begin{aligned}
      \lim_{\beta\rightarrow\infty}\mu=\varepsilon_{F}
    \end{aligned}$, 对应波矢 $|k|\leq k_{F}$. 由于前面讨论 $k_{i}\in\mathbb{N}^{*}$, 因此 $k\leq k_{F}$ 在 $\vec{k}$ 空间中会形成 $\begin{aligned}
      \frac{1}{8}
    \end{aligned}$ 球体. 由于题解要求, 我们略去讨论各原子贡献的自由电子数目, 而是直接使用总粒子(电子)数 $N$: 
    \begin{align*}
      \frac{1}{8}\left(\frac{4}{3}\pi k_{F}^{3}\right) = \frac{N}{2}\left(\frac{\pi^{3}}{V}\right)
    \end{align*}
    其中 $N$ 除以 $2$ 是因为泡利不相容原理. 具体到题目中, 有 $l_{i}=L,\forall i$, 于是进一步化简得到
    \begin{align*}
      \boxed{N = \frac{k_{F}^{3}V}{3\pi^{2}}},\quad \frac{N}{V} = \boxed{n = \frac{k_{F}^{3}}{3\pi^{2}}}
    \end{align*}
    接下来计算总能量. 假设 $N$ 充分大, 使得电子可存在的状态遍布整个半径为 $k_{F}$ 的 $\begin{aligned}
      \frac{1}{8}
    \end{aligned}$ 费米球, 于是求和化为积分形式, 即有
    %%%%
    $\begin{aligned}
      E_{\text{tot}} = \sum_{i}^{k\leq k_{F}}\frac{\hbar^{2}k^{2}}{2m}\Rightarrow \int_{0}^{k_{F}}\frac{\hbar^{2}k^{2}}{2m}f(k)\mathrm{d}k
    \end{aligned}$, 其中 $f(k)$ 是态密度, 表示在同一能量 $\begin{aligned}
      \frac{\hbar^{2}k^{2}}{2m}
    \end{aligned}$ 上的电子数目, 所以这就要求我们对电子态密度进行计算. 对于半径为 $k$, 厚度为 $\mathrm{d}k$ 的 $\begin{aligned}
      \frac{1}{8}
    \end{aligned}$ 球壳, 在这个球壳上电子的能量都是相同的. 而这个球壳的体积为 $\begin{aligned}
      \frac{1}{8} (4\pi k^{2}\mathrm{d}k)
    \end{aligned}$, 又已知每个格点体积为 $\begin{aligned}
      \frac{\pi^{3}}{V}
    \end{aligned}$, 因此球壳中电子数目为
    \begin{align*}
      \text{格点数}\times 2 = \frac{\frac{1}{8} (4\pi k^{2}\mathrm{d}k)}{\frac{\pi^{3}}{V}}\times 2 = \frac{ k^{2}V}{\pi^{2}}\mathrm{d}k = f(k)\mathrm{d}k
    \end{align*}
    因此总能量为
    \begin{align*}
      E^{(0)} = \int_{0}^{k_{F}}\frac{\hbar^{2}k^{2}}{2m}\frac{k^{2}V}{\pi^{2}}\mathrm{d}k = \frac{\hbar^{2}V}{2m\pi^{2}}\int_{0}^{k_{F}}k^{4}\mathrm{d}k = \frac{\hbar^{2}V}{2m\pi^{2}}\frac{k_{F}^{5}}{5} = \boxed{\frac{\hbar^{2}Vk_{F}^{5}}{10m\pi^{2}}}
    \end{align*}
    反解粒子数密度表达式得到 $k_{F}(n)$, 代入 $E^{(0)}$ 计算总能量密度:
    \begin{align*}
      k_{F} &= (3\pi^{2}n)^{\frac{1}{3}}\\
      \frac{E^{(0)}}{V} &= \frac{\hbar^{2}k_{F}^{5}}{10m\pi^{2}} = \frac{\hbar^{2}}{10m\pi^{2}}\cdot (3\pi^{2}n)^{\frac{5}{3}}=\boxed{\frac{(3n)^{\frac{5}{3}}\hbar^{2}\pi^{\frac{4}{3}}}{10m}}
    \end{align*}}}
    
    \item \textbf{计算能量的一阶修正 $E^{(1)} = \langle\text{FS}|H_{I}|\text{FS}\rangle$.}
    
    题目中定义的傅里叶变换是非幺正的, 代入结论的时候需要注意系数. 

    \begin{align*}
      v(\vec{q}) = \frac{1}{V}\int\frac{1}{|\vec{x}|}e^{i\vec{q}\cdot\vec{x}}\mathrm{d}\vec{x} = \frac{1}{V}\frac{4\pi}{q^{2}}
    \end{align*}

    代 $v(\vec{q})$ 入两体相互作用部分, 有

    \begin{align*}
      H_{I} = \frac{1}{2V}
      \sum_{\vec{k}_{1},\vec{k}_{2},\vec{q}}
      \sum_{\sigma,\sigma^{\prime}}
      \frac{1}{V}\frac{4\pi}{q^{2}}
      c^{\dagger}_{\vec{k}_{1}+\vec{q},\sigma}
      c^{\dagger}_{\vec{k}_{2}-\vec{q},\sigma^{\prime}}
      c_{\vec{k}_{2},\sigma^{\prime}}
      c_{\vec{k}_{1},\sigma}
    \end{align*}
    
    \item \textbf{利用 Hatree Fock 平均场近似, 并假设平均场参数是自旋对角的, 并且保持了自旋对称性, 以及平移对称性, 因此我们期待 $\left\langle c_{\vec{k}\sigma}^{\dagger}c_{\vec{k}^{\prime}\sigma^{\prime}}\right\rangle = \left\langle c_{\vec{k}\sigma}^{\dagger}c_{\vec{k}\sigma}\right\rangle\delta_{\vec{k},\vec{k}^{\prime}}\delta_{\sigma,\sigma^{\prime}}$, 以及 $\left\langle c_{\vec{k}\uparrow}^{\dagger}c_{\vec{k}\uparrow}\right\rangle = \left\langle c_{\vec{k}\downarrow}^{\dagger}c_{\vec{k}\downarrow}\right\rangle$. 计算系统总能量, 并与 $E^{(0)} + E^{(1)}$ 比较大小.}
    
    代 $\begin{aligned}
      |\text{HF}\rangle = \prod_{k\leq k_{F},\sigma}c_{\vec{k},\sigma}^{\dagger}|0\rangle
    \end{aligned}$ 入能量一阶修正, 有

    \begin{align*}
      \langle\text{HF}|H_{0}|\text{HF}\rangle &= \sum_{\vec{k},\sigma}\langle\text{HF}|\frac{k^{2}}{2}c_{\vec{k},\sigma}^{\dagger}c_{\vec{k},\sigma}|\text{HF}\rangle
    \end{align*}

    \begin{align*}
      \langle\text{HF}|H_{I}|\text{HF}\rangle &= \frac{1}{2V}\frac{4\pi}{V}\sum_{\vec{k}_{1},\vec{k}_{2},\vec{q}}\sum_{\sigma,\sigma^{\prime}}\frac{1}{q^{2}}
      \langle\text{HF}|\underbrace{
      c^{\dagger}_{\vec{k}_{1}+\vec{q},\sigma} % lambda
      c^{\dagger}_{\vec{k}_{2}-\vec{q},\sigma^{\prime}} % mu
      c_{\vec{k}_{2},\sigma^{\prime}} %rho
      c_{\vec{k}_{1},\sigma} %nu
      }_{c_{\lambda}^{\dagger}c_{\mu}^{\dagger}c_{\rho}c_{\nu}}
      |\text{HF}\rangle\\ 
      &= \frac{1}{2V}\frac{4\pi}{V}\sum_{\vec{k}_{1},\vec{k}_{2},\vec{q}}\sum_{\sigma,\sigma^{\prime}}\frac{1}{q^{2}}
(\cancel{\delta_{\vec{k}_{1}+\vec{q},\vec{k}_{1}}
\delta_{\vec{k}_{2}-\vec{q},\vec{k}_{2}}} - 
\delta_{\vec{k}_{1}+\vec{q},\vec{k}_{2}}\delta_{\sigma,\sigma^{\prime}}
\delta_{\vec{k}_{2}-\vec{q},\vec{k}_{1}}\delta_{\sigma^{\prime},\sigma}),\quad v(\vec{q}=0)\text{不发散}\\
      &= -\frac{1}{2V}\frac{4\pi}{V}
      \sum_{\vec{k}_{1}}
      \sum_{\vec{k}_{2}}
      \sum_{\vec{q}}
      \sum_{\sigma}
      \sum_{\sigma^{\prime}}
      \frac{1}{q^{2}}
      \delta_{\vec{k}_{1}+\vec{q},\vec{k}_{2}}
      \delta_{\vec{k}_{2}-\vec{q},\vec{k}_{1}}
      \delta_{\sigma^{\prime},\sigma}
      \delta_{\sigma,\sigma^{\prime}}\\
      &=-\frac{1}{2V}\frac{4\pi}{V}
      \sum_{\vec{k}_{1}}
      \sum_{\vec{k}_{2}}
      \sum_{\vec{q}}
      \sum_{\sigma}
      \frac{1}{q^{2}}
      \delta_{\vec{k}_{1}+\vec{q},\vec{k}_{2}}
      \delta_{\vec{k}_{2}-\vec{q},\vec{k}_{1}}\\
      &= -\frac{1}{V}\frac{4\pi}{V}
      \sum_{\vec{k}_{1}}
      \sum_{\vec{k}_{2}}
      \sum_{\vec{q}}
      \frac{1}{q^{2}}
      \delta_{\vec{k}_{1}+\vec{q},\vec{k}_{2}}
      \delta_{\vec{k}_{2}-\vec{q},\vec{k}_{1}}\\
      &= -\frac{1}{V}\frac{4\pi}{V}
      \sum_{\vec{k}_{1}}
      \sum_{\vec{k}_{2}}
      \int\mathrm{d}\vec{q}\frac{V}{(2\pi)^{3}}
      \frac{1}{q^{2}}
      \delta_{\vec{q},\vec{k}_{2}-\vec{k}_{1}}
      \delta_{\vec{q},\vec{k}_{2}-\vec{k}_{1}}\\
      &= -\frac{1}{V}
      \sum_{\vec{k}_{1}}
      \sum_{\vec{k}_{2}}
      \frac{4\pi}{|\vec{k}_{1}-\vec{k}_{2}|^{2}}
    \end{align*}

    在第二行消去了一项, 这是因为它会引起 $\vec{q} = 0$. 有关于最后一行的求和, 这是一个固定结论, 没有必要在考场现场计算求和, 在这里直接给出答案:

    \begin{align*}
      \langle\text{HF}|H_{I}|\text{HF}\rangle = -\frac{k_{F}^{3}V}{4\pi^{3}} &= -\frac{3}{4}\left(\frac{3}{\pi}\right)^{\frac{1}{3}}n^{\frac{4}{3}}V\\
      \Rightarrow E &= \frac{(3n)^{\frac{5}{3}}\pi^{\frac{4}{3}}V}{10} - \frac{3}{4}\left(\frac{3}{\pi}\right)^{\frac{1}{3}}n^{\frac{4}{3}}V
    \end{align*}
  \end{enumerate}

  \item \textbf{量子转子模型}
  
  \textbf{量子转子的角度坐标 $\theta\in[0,2\pi)$, 注意 $\theta\pm 2\pi$ 和 $\theta$ 是等价的. 用 $|\theta\rangle$ 表现 $\hat{\theta}$ 算符的本征态, $|\theta\pm 2\pi\rangle$ 和 $|\theta\rangle$ 是相同的态. 定义量子转子的转动算符为 $\hat{R}(\alpha)$, 
  \begin{align*}
    \hat{R}(\alpha) = \int_{0}^{2\pi}\mathrm{d}\theta |\theta - \alpha\rangle\langle\theta|
  \end{align*}
  所以 $\hat{R}(\alpha)|\theta\rangle = |\theta - \alpha\rangle$, 并且 $\hat{R}(2\pi)$ 是单位算符.}

  \textbf{转动算符 $\hat{R}S(\alpha)$ 是一个幺正算符, 它的产生子为厄米算符 $\hat{N}$, 与量子转子的角动量算符 $\hat{L}$ 的关系为 $\hat{L} = \hbar\hat{N}$, 所以 $\hat{R}(\alpha) = e^{i\hat{N}\alpha}$, 在 $\hat{\theta}$ 表象下可求得 $\hat{N} = -i\frac{\partial}{\partial\theta}$. }

  \textbf{考虑一个特定的量子转子模型, 它的哈密顿量为
  \begin{align*}
    H = \frac{1}{2}\left(\hat{N} - \frac{1}{2}\right)^{2} - g\cos{\left(2\hat{\theta}\right)}
  \end{align*}
  其中 $g\cos{\left(2\hat{\theta}\right)}$ 是一个小的外势, 可以当成微扰处理. 假设 $|N\rangle$ 是算符 $\hat{N}$ 的本征态, 本征值为 $N$, 即 $\hat{N}|N\rangle = N|N\rangle$. 可计算出 $|N\rangle$ 用 $|\theta\rangle$ 展开为
  \begin{align*}
    |N\rangle = \frac{1}{\sqrt{2\pi}}\int_{0}^{2\pi}e^{iN\theta}|\theta\rangle
  \end{align*}}
  \begin{enumerate}
    \item \textbf{利用 $\hat{R}(2\pi)$ 是单位算符证明 $N$ 必须是整数.}
    
    {\color{gray}{因为 $\hat{R}(2\pi) = \mathbb{I}$, 所以有 $|\theta - 2\pi\rangle = |\theta\rangle$. 对于算符 $\hat{N}$ 的本征态 $|N\rangle$ 有
  \begin{align*}    
    \frac{1}{\sqrt{2\pi}}\int_{0}^{2\pi}\mathrm{d}\theta e^{iN(\theta - 2\pi)}|\theta-2\pi\rangle &= \frac{1}{\sqrt{2\pi}}\int_{0}^{2\pi}\mathrm{d}\theta e^{iN\theta}|\theta\rangle\\
    \iff \frac{1}{\sqrt{2\pi}}\int_{0}^{2\pi}\mathrm{d}\theta e^{iN(\theta - 2\pi)}|\theta\rangle &= \frac{1}{\sqrt{2\pi}}\int_{0}^{2\pi}\mathrm{d}\theta e^{iN(\theta - 2\pi)}|\theta\rangle\\
    \iff e^{iN\theta}  &= e^{iN(\theta - 2\pi)} = e^{iN\theta}e^{-i2\pi N}
  \end{align*}
  因此为了保持 $\theta$ 转动 $2\pi$ 后的不变性, $N$ 应当是整数.}}
    
    \item \textbf{考虑无微扰时的哈密顿量 $H_{0} = \frac{1}{2}\left(\hat{N} - \frac{1}{2}\right)^{2}$, 证明 $|N\rangle$ 也是 $H_{0}$ 的本征态, 并求出本征能量, 证明每个能级都是两重简并的. }
    
    {\color{gray}{  \begin{align*}
      \hat{H}_{0}|N\rangle = \frac{1}{2}\left(\hat{N} - \frac{1}{2}\right)^{2}|N\rangle = \frac{1}{2}\left(N - \frac{1}{2}\right)^{2}|N\rangle\Rightarrow E_{N}^{(0)} = \frac{1}{2}\left(N - \frac{1}{2}\right)^{2}\\
      \Rightarrow N_{\pm} - \frac{1}{2} = \pm\sqrt{2E_{N}^{(0)}}\Rightarrow N_{\pm} = \frac{1}{2} \pm \sqrt{2E_{N}^{(0)}}
    \end{align*}
    这意味着对于任意整数 $N$, 都对应存在着 $N^{\prime} = 1 - N$ 使得能级简并.}}
    
    \item \textbf{采用 $\{|N\rangle\}$ 作为基组, 写出微扰项 $V = -g\cos{\left(2\hat{\theta}\right)}$ 的表示矩阵, 并证明微扰不会连接简并的能级(即如果 $|N\rangle$ 和 $|N^{\prime}\rangle$ 简并, 那么 $\langle N|V|N^{\prime}\rangle$). 因此尽管 $H_{0}$ 的能级是简并的, 我们仍然可以使用非简并微扰论.}
    
    {\color{gray}{  \begin{align*}
      \cos{2\hat{\theta}} &= \frac{1}{2}\left(e^{i2\hat{\theta}} + e^{-i2\hat{\theta}}\right)\\
      e^{i2\hat{\theta}}|N\rangle &= e^{i2\hat{\theta}}\left(\frac{1}{\sqrt{2\pi}}\int_{0}^{2\pi}\mathrm{d}\theta e^{iN\theta}|\theta\rangle\right) = \frac{1}{\sqrt{2\pi}}\int_{0}^{2\pi}\mathrm{d}\theta e^{iN\theta}e^{i2\hat{\theta}}|\theta\rangle \\
      &= \frac{1}{\sqrt{2\pi}}\int_{0}^{2\pi}\mathrm{d}\theta e^{i(N+2)\theta}|\theta\rangle = |N+2\rangle\\
      \Rightarrow \cos{2\hat{\theta}}|N\rangle &= \frac{1}{2}\left(e^{i2\hat{\theta}} + e^{-i2\hat{\theta}}\right)|N\rangle = \frac{1}{2}\left(|N+2\rangle + |N-2\rangle\right)\\
      \Rightarrow \langle N|\hat{V}|N^{\prime}\rangle &= -g\langle N|\cos{2\hat{\theta}}|N^{\prime}\rangle = -\frac{g}{2}\left(\langle N|N^{\prime}+2\rangle + \langle N|N^{\prime}-2\rangle\right)\\
      & = -\frac{g}{2}(\delta_{N,N^{\prime}+2} + \delta_{N,N^{\prime}-2})
    \end{align*}
    和前文一致, 如果 $|N\rangle$ 和 $|N^{\prime}\rangle$ 简并, 那么 $N + N^{\prime} = 1$ 使得只要 $N\in\mathbb{Z}$, 那么$\delta\neq 0$. 所以仍然可以使用非简并微扰论.}}
    
    \item \textbf{计算每个能级 $E_{N}$ 的微扰修正到 $g$ 的二阶, 并证明此时所有的能级简并仍然没有被解除.}
    
    {\color{gray}{  \begin{align*}
      E_{N}^{(1)} &= \langle N|\hat{V}|N\rangle = -\frac{g}{2}\left(\langle N|N+2\rangle + \langle N|N-2\rangle\right) = 0\\
      E_{N}^{(2)} &= \sum_{N^{\prime}\neq N}\frac{|\langle N|\hat{V}|N^{\prime}\rangle|^{2}}{E_{N}^{(0)} - E_{N^{\prime}}^{(0)}} = \sum_{N^{\prime}\neq N}\frac{\left(-\frac{g}{2}(\delta_{N,N^{\prime}+2} + \delta_{N, N^{\prime}-2})\right)^{2}}{\frac{1}{2}\left(N - \frac{1}{2}\right)^{2} - \frac{1}{2}\left(N^{\prime} - \frac{1}{2}\right)^{2}}\\
      &= \boxed{\frac{g^{2}}{(2N-3)(2N+1)}}
    \end{align*}
    微扰修正后的能级为
    \begin{align*}
      E_{N} \approx \frac{1}{2}\left(N - \frac{1}{2}\right)^{2} + \frac{g^{2}}{(2N-3)(2N+1)}
    \end{align*}
    代入 $N^{\prime} = 1 - N$ 以检查能级简并性:
    \begin{align*}
      E_{N^{\prime}} &= \frac{1}{2}\left(1 - N - \frac{1}{2}\right)^{2} + \frac{g^{2}}{[2(1 - N)-3][2(1 - N)+1]}\\
      &= \frac{1}{2}\left(N - \frac{1}{2}\right)^{2} + \frac{g^{2}}{(2N+1)(2N-3)} = E_{N}
    \end{align*}
    所以简并度未变化.}}
  \end{enumerate}
\end{enumerate}

\end{document}