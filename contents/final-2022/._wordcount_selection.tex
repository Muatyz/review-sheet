\begin{align*}
%       H_{0}|s,m\rangle &= \left(S^{z}\right)^{2}|s,m\rangle = m^{2}|s,m\rangle\\
%       \Rightarrow E_{-1}^{(0)} & = 1,\quad E_{0} = 0,\quad E_{1} = 1
%     \end{align*}
%     注意到 $m^{2}$ 会带来 $m=\pm 1$ 的简并, 所以后续计算时会涉及简并态的微扰处理. 首先观察简并态, 简并态矢张成独立子空间, 于是求解这个子空间中 $V$ 的矩阵:
%     \begin{align*}
%       V_{\text{sub}} &= \begin{pmatrix}
%         \langle 1,1|V|1,1\rangle & \langle 1,1|V|1,-1\rangle\\
%         \langle 1,-1|V|1,1\rangle & \langle 1,-1|V|1,-1\rangle
%       \end{pmatrix}\\
%       \langle 1,1|V|1,1\rangle &= \begin{pmatrix}
%         1 & 0 & 0
%       \end{pmatrix}\begin{pmatrix}
%         1 & \frac{1}{\sqrt{2}}& \\
%         \frac{1}{\sqrt{2}} & 0  & \frac{1}{\sqrt{2}}\\
%           & \frac{1}{\sqrt{2}} & -1
%        \end{pmatrix}\begin{pmatrix}
%           1\\
%           0\\
%           0
%        \end{pmatrix} = 1,\\
%        \langle 1,1|V|1,-1\rangle &= \begin{pmatrix}
%         1 & 0 & 0
%       \end{pmatrix}\begin{pmatrix}
%         1 & \frac{1}{\sqrt{2}}& \\
%         \frac{1}{\sqrt{2}} & 0  & \frac{1}{\sqrt{2}}\\
%           & \frac{1}{\sqrt{2}} & -1
%        \end{pmatrix}\begin{pmatrix}
%           0\\
%           0\\
%           1
%         \end{pmatrix} = 0,\\
%         \langle 1,-1|V|1,1\rangle &= 0,\\
%         \langle 1,-1|V|1,-1\rangle &= \begin{pmatrix}
%           0 & 0 & 1
%         \end{pmatrix}\begin{pmatrix}
%           1 & \frac{1}{\sqrt{2}}& \\
%           \frac{1}{\sqrt{2}} & 0  & \frac{1}{\sqrt{2}}\\
%             & \frac{1}{\sqrt{2}} & -1
%          \end{pmatrix}\begin{pmatrix}
%             0\\
%             0\\
%             1
%          \end{pmatrix} = -1.\\
%          \Rightarrow V_{\text{sub}} &= \begin{pmatrix}
%           1 & 0\\
%           0 & -1
%          \end{pmatrix}
%     \end{align*}
%     注意到计算得到的子空间中 $V_{\text{sub}}$ 完成了对角化, 这说明沿用的 $|s,m\rangle$ 基矢已经是 "好量子态". 所以回归到非简并微扰论的方法. 一阶能量修正各为
%     \begin{align*}
%       E_{1}^{(1)} &= \langle 1,1|V|1,1\rangle = \begin{pmatrix}
%         1 & 0 & 0
%       \end{pmatrix}\begin{pmatrix}
%         1 & \frac{1}{\sqrt{2}}& \\
%         \frac{1}{\sqrt{2}} & 0  & \frac{1}{\sqrt{2}}\\
%           & \frac{1}{\sqrt{2}} & -1
%        \end{pmatrix}\begin{pmatrix}
%           1\\
%           0\\
%           0
%        \end{pmatrix} = \boxed{1},\\
%        E_{0}^{(1)} &= \langle 1,0|V|1,0\rangle = \begin{pmatrix}
%         0 & 1 & 0
%       \end{pmatrix}\begin{pmatrix}
%         1 & \frac{1}{\sqrt{2}}& \\
%         \frac{1}{\sqrt{2}} & 0  & \frac{1}{\sqrt{2}}\\
%           & \frac{1}{\sqrt{2}} & -1
%        \end{pmatrix}\begin{pmatrix}
%           0\\
%           1\\
%           0
%        \end{pmatrix} = \boxed{0},\\
%        E_{-1}^{(1)} &= \langle 1,-1|V|1,-1\rangle = \begin{pmatrix}
%         0 & 0 & 1
%       \end{pmatrix}\begin{pmatrix}
%         1 & \frac{1}{\sqrt{2}}& \\
%         \frac{1}{\sqrt{2}} & 0  & \frac{1}{\sqrt{2}}\\
%           & \frac{1}{\sqrt{2}} & -1
%        \end{pmatrix}\begin{pmatrix}
%           0\\
%           0\\
%           1
%        \end{pmatrix} = \boxed{-1},\\
%     \end{align*}
%     二阶能量修正由公式 $\begin{aligned}
%       E_{m}^{(n)} = \sum_{n\neq m}\frac{|\langle n|V|m\rangle|^{2}}{E_{m}^{(0)} - E_{n}^{(0)}}
%     \end{aligned}$ 给出:
%     \begin{align*}
%       E_{1}^{(2)} &= \frac{|\langle 1,0|V|1,1\rangle|^{2}}{E_{1}^{(0)}-E_{0}^{0}} + \frac{|\langle 1,-1|V|1,1\rangle|^{2}}{E_{1}^{(0)} - E_{-1}^{(0)}} = \frac{\left(\frac{1}{\sqrt{2}}\right)^{2}}{1 - 0} + \frac{0^{2}}{1 - 1} = \boxed{\frac{1}{2}},\\
%       E_{0}^{(2)} &= \frac{|\langle 1,1|V|1,0\rangle|^{2}}{E_{0}^{(0)} - E_{1}^{(0)}} + \frac{|\langle 1,-1|V|1,0\rangle|^{2}}{E_{0}^{(0)} - E_{-1}^{(0)}} = \frac{\left(\frac{1}{\sqrt{2}}\right)^{2}}{0-1} + \frac{0^{2}}{0 - 1} = \boxed{-\frac{1}{2}},\\
%       E_{-1}^{(2)} &= \frac{|\langle 1,0|V|1,-1\rangle|^{2}}{E_{-1}^{(0)} - E_{0}^{(0)}} + \frac{|\langle 1,1|V|1,-1\rangle|^{2}}{E_{-1}^{(0)} - E_{1}^{(0)}} = \frac{\left(\frac{1}{\sqrt{2}}\right)^{2}}{1-0} + \frac{0^{2}}{1 - 1} = \boxed{\frac{1}{2}}
%     \end{align*}
%     可见, 只要在 $E_{i}^{(1)} - E_{j}^{(1)}=0$ 时分子也为 $0$, 我们就可以无视分母为 $0$ 的问题. 
%     接下来是对态函数的微扰修正. 一阶修正由 $\begin{aligned}
%       |m\rangle^{(1)} = \sum_{n\neq m}|n\rangle\frac{\langle n|V|m\rangle}{E_{m}^{(0)} - E_{n}^{(0)}}
%     \end{aligned}$ 给出:
%     \begin{align*}
%       |1,1\rangle^{(1)} &= |1,0\rangle\frac{\langle 1,0|V|1,1\rangle}{E_{1}^{(0)} - E_{0}^{(0)}} + |1,-1\rangle\frac{\langle 1,-1|V|1,1\rangle}{E_{1}^{(0)} - E_{-1}^{(0)}} = |1,0\rangle\frac{1}{\sqrt{2}}\frac{1}{1 - 0} + |1,-1\rangle\cdot 0 \\
%       &= \boxed{\frac{1}{\sqrt{2}}|1,0\rangle}\\
%       |1,0\rangle^{(1)} &= |1,1\rangle\frac{\langle 1,1|V|1,0\rangle}{E_{0}^{(0)} - E_{1}^{(0)}} + |1,-1\rangle\frac{\langle 1,-1|V|1,0\rangle}{E_{0}^{(0)} - E_{-1}^{(0)}} = |1,1\rangle \frac{1}{\sqrt{2}}\frac{1}{0-1} + |1,-1\rangle \frac{1}{\sqrt{2}}\cdot\frac{1}{0-1} \\
%       &= \boxed{-\frac{1}{\sqrt{2}}(|1,1\rangle + |1,-1\rangle)}\\
%       |1,-1\rangle^{(1)} &= |1,1\rangle\frac{\langle 1,1|V|1,-1\rangle}{E_{-1}^{(0)}-E_{1}^{(0)}} + |1,0\rangle\frac{\langle 1,0|V|1,-1\rangle}{E_{-1}^{(0)}-E_{0}^{(0)}} = |1,1\rangle\cdot 0 + |1,0\rangle\frac{1}{\sqrt{2}}\cdot\frac{1}{1-0} \\
%       &= \boxed{\frac{1}{\sqrt{2}}|1,0\rangle}
%     \end{align*}
%     总结:
%     \begin{align*}
%       E_{1} & = 1 + 1\lambda + \frac{1}{2}\lambda^{2} + \mathit{o}(\lambda^{2})\\
%       E_{0} & = 0 + 0\lambda - \frac{1}{2}\lambda^{2} +  \mathit{o}(\lambda^{2})\\
%       E_{-1} &= 1 - 1\lambda + \frac{1}{2}\lambda^{2} + \mathit{o}(\lambda^{2})\\
%       |1,1\rangle &= |1,1\rangle + \frac{\lambda}{\sqrt{2}}|1,0\rangle + \mathit{o}(\lambda)\\
%       |1,0\rangle &= |1,0\rangle - \frac{\lambda}{\sqrt{2}}(|1,1\rangle + |1,-1\rangle) + \mathit{o}(\lambda)\\
%       |1,-1\rangle &= |1,-1\rangle + \frac{\lambda}{\sqrt{2}}|1,0\rangle + \mathit{o}(\lambda)
%     \end{align*}
%     对于这类可以使用矩阵形式讨论的问题, 还有一种笨办法, 就是直接严格对角化含 $\lambda$ 微扰的哈密顿量, 然后进行 Taylor 展开得到各级数. 但是在三阶矩阵下的计算已经非常复杂, 所以还是建议使用一般微扰论方法, 毕竟考试时是会给出公式的.