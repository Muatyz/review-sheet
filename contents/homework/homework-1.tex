\documentclass[../../main.tex]{subfiles}
\graphicspath{{\subfix{../images/}}} % 指定图片目录,后续可以直接使用图片文件名。
\begin{document}
\section{Homework 1}
\subsection{Hermitian operators}

\begin{enumerate}
  \item  \textbf{Prove theorem 1: If $A$ is Hermitian operator, then all its eigenvalues are real numbers, and the eigenvectors corresponding to different eigenvalues are orthogonal.}
{\color{gray}{  \begin{enumerate}
    \item Since $A$ is Hermitian, we have $A^{\dagger} = A$. Let $\lambda$ be an eigenvalue of $A$ and $v$ the corresponding eigenvector, so
    \begin{align*}
      A v = \lambda v.
    \end{align*}
    Consider the inner product
    \begin{align*}
      \langle v, Av\rangle &= \langle v,\lambda v\rangle = \lambda \langle v, v\rangle = \lambda||v||^{2}.\\
      \langle Av,v\rangle &= \langle \lambda v, v\rangle = \lambda^{*} \langle v, v\rangle = \lambda^{*}||v||^{2}.
    \end{align*}
    So we have $\lambda||v||^{2} = \lambda^{*}||v||^{2}$, which implies $\lambda = \lambda^{*}$, so $\lambda$ is real(since $||v||^{2}$ is not zero, as $v\neq 0$).

    \item Let $\lambda_{1}$ and $\lambda_{2}$ be two different eigenvalues of $A$, and $v_{1}$ and $v_{2}$ the corresponding eigenvectors, so we have
    \begin{align*}
      A v_{1} = \lambda_{1} v_{1},\quad A v_{2} = \lambda_{2} v_{2}.
    \end{align*}
    Consider the inner product
    \begin{align*}
      \langle v_{1}, Av_{2}\rangle = \langle v_{1}, \lambda_{2} v_{2}\rangle = \lambda_{2}\langle v_{1}, v_{2}\rangle,\\
      \langle Av_{1}, v_{2}\rangle = \langle \lambda_{1} v_{1}, v_{2}\rangle = \lambda_{1}\langle v_{1}, v_{2}\rangle.
    \end{align*}
    Since $A$ is Hermitian, we have $\langle v_{1}, Av_{2}\rangle = \langle Av_{1}, v_{2}\rangle$, so we have $(\lambda_{1}-\lambda_{2})\langle v_{1}, v_{2}\rangle = 0$, which implies $\langle v_{1}, v_{2}\rangle = 0$(since $\lambda_{1}\neq \lambda_{2}$).$\square$
  \end{enumerate}}}
  \item \textbf{Prove theorem 2: If $A$ is Hermitian operator, then it can be always diagonalized by unitary transformation.}
  
  {\color{gray}{Let $\{\lambda_{1}, \lambda_{2},\cdots, \lambda_{n}\}$ be the eigenvalues of $A$, and $\{v_{1}, v_{2},\cdots,v_{n}\}$ the corresponding eigenvectors. 
  
  By theorem 1, we have $\langle v_{1}, v_{2}\rangle = \delta_{ij}$. 

  We define the unitary matrix as $U = [v_{1}, v_{2},\cdots,v_{n}]$, so we have $U^{\dagger}U = \mathbb{I}$. Now we compute $U^{\dagger}AU$. Since $Av_{i}=\lambda_{i}v_{i}$, we have
  \begin{align*}
    U^{\dagger}AU &= \begin{pmatrix}
      v_{1}^{\dagger} \\
      v_{2}^{\dagger} \\
      \vdots \\
      v_{n}^{\dagger}
    \end{pmatrix}A\begin{pmatrix}
      v_{1} & v_{2} & \cdots & v_{n}
    \end{pmatrix} = \begin{pmatrix}
      v_{1}^{\dagger}Av_{1} & v_{1}^{\dagger}Av_{2} & \cdots & v_{1}^{\dagger}Av_{n} \\
      v_{2}^{\dagger}Av_{1} & v_{2}^{\dagger}Av_{2} & \cdots & v_{2}^{\dagger}Av_{n} \\
      \vdots & \vdots & \ddots & \vdots \\
      v_{n}^{\dagger}Av_{1} & v_{n}^{\dagger}Av_{2} & \cdots & v_{n}^{\dagger}Av_{n}
    \end{pmatrix} \\
    &= \begin{pmatrix}
      \lambda_{1} & 0 & \cdots & 0 \\
      0 & \lambda_{2} & \cdots & 0 \\
      \vdots & \vdots & \ddots & \vdots \\
      0 & 0 & \cdots & \lambda_{n}
    \end{pmatrix} = \Lambda.\square
  \end{align*}}}
  \item  \textbf{Prove theorem 3: Two diagonalizable operators $A$ and $B$ can be simultaneously diagonalized if, and only if, $[A,B]=0$.}
  {\color{gray}{\begin{enumerate}
    \item Let's say 
    \begin{align*}
      A|v\rangle = \lambda|v\rangle, \quad B|v\rangle = \mu|v\rangle.
    \end{align*}
    where $|v\rangle$ is the eigenvector of $A$ and $B$, $\lambda$ and $\mu$ are the corresponding eigenvalues.

    So 
    \begin{align*}
      [A,B]|v\rangle = (AB-BA)|v\rangle = (AB|v\rangle - BA|v\rangle) = (\lambda\mu - \mu\lambda)|v\rangle = 0.
    \end{align*}
    for all $|v\rangle$, which means $[A,B]=0$.
    \item Let's say $[A,B]=0$. And we have 
    \begin{align*}
      A|v\rangle &= \lambda|v\rangle,\\
      AB|v\rangle &= BA|v\rangle = B\lambda|v\rangle = \lambda \left(B|v\rangle\right),
    \end{align*}
    which means $B|v\rangle$ is also the eigenvector of $A$ with eigenvalue $\lambda$. And apply the same method to all $|v\rangle$ of $A$, we can find a common set of eigenvectors of $A$ and $B$ within the degenerate subspace. $\square$
  \end{enumerate}}}
\end{enumerate}

\subsection{Matrix diagonalization and unitary transformation}
\begin{enumerate}
  \item  \textbf{Diagonalizing a matrix $L$ corresponds to finding a unitary transformation $V$ such that $L=V\Lambda V^{\dagger}$, where $\Lambda$ is a diagonal matrix whose diagonal elements are eigenvalues, $V$ is an unitary matrix whose column vectors are the corresponding eigenstates. Find a unitary matrix $V$ that can diagonalize the Pauli matrix $\sigma^{x}_{(z)} = \begin{pmatrix}
    0 & 1 \\
    1 & 0\end{pmatrix}$, and find the eigenvalues of $\sigma^{x}_{(z)}$.}

    {\color{gray}{Find the eigenvalues of $\sigma^{x}_{(z)}$ by solving the characteristic equation
    \begin{align*}
      \text{det}(\sigma^{x}_{(z)}-\lambda I) = \text{det}\begin{pmatrix}
        -\lambda & 1 \\
        1 & -\lambda
      \end{pmatrix} = \lambda^{2} - 1 = 0,
    \end{align*}
    So we have $\boxed{\lambda = \pm 1}$. For $\lambda_{+} = 1$, we have 
    \begin{align*}
      \begin{pmatrix}
        0 & 1 \\
        1 & 0
      \end{pmatrix}
      \begin{pmatrix}
        v_{1}\\
        v_{2}
      \end{pmatrix}
      = 1\cdot\begin{pmatrix}
        v_{1}\\
        v_{2}
      \end{pmatrix}\Rightarrow v_{1} = v_{2}.
    \end{align*}
    So the eigenvector corresponding to $\lambda_{+}$ is $\begin{aligned}
      |+\rangle^{x}_{(z)} = \frac{1}{\sqrt{2}}\begin{pmatrix}
        1\\
        1
      \end{pmatrix}
    \end{aligned}$. For $\lambda_{-} = -1$, we have
    \begin{align*}
      \begin{pmatrix}
        0 & 1 \\
        1 & 0
      \end{pmatrix}
      \begin{pmatrix}
        v_{1}\\
        v_{2}
      \end{pmatrix}
      = -1\cdot\begin{pmatrix}
        v_{1}\\
        v_{2}
      \end{pmatrix}\Rightarrow v_{1} = -v_{2}.
    \end{align*}
    So the eigenvector corresponding to $\lambda_{-}$ is $\begin{aligned}
      |-\rangle^{x}_{(z)} = \frac{1}{\sqrt{2}}\begin{pmatrix}
        1\\
        -1
      \end{pmatrix}
    \end{aligned}$. The eigenvectors have been normalized, so the unitary matrix $V$ is $\begin{aligned}
      [|+\rangle^{x}_{(z)},|-\rangle^{x}_{(z)}] = \frac{1}{\sqrt{2}}\begin{pmatrix}
        1 & 1 \\
        1 & -1
      \end{pmatrix}
    \end{aligned}$.
    The diagonal matrix $\Lambda$ contains the eigenvalues on the diagonal, which means
    \begin{align*}
      \Lambda = \text{diag}\{\lambda_{+},\lambda_{-}\} =\begin{pmatrix}
        1 & 0 \\
        0 & -1
      \end{pmatrix} = \sigma^{z}_{(z)}
    \end{align*}
    Thus we diagonalized the Pauli matrix $\sigma^{x}_{(z)}$ by the unitary transformation $V$:
    \begin{align*}
      \sigma^{x}_{(z)} = V^{\dagger}\Lambda V = V^{\dagger}\sigma^{z}_{(z)} V
    \end{align*}
    We notice that the diagnosed matrix $\Lambda$ is just the Pauli matrix $\sigma^{z}_{(z)}$, which means we can transform the representation of the Pauli matrix $\sigma^{z}$ to the $\sigma^{x}$ representation by the unitary transformation $V$:
    \begin{align*}
      \sigma^{x}_{(z)} = V^{\dagger}\sigma^{z}_{(z)} V = V^{\dagger}\sigma^{x}_{(x)} V\Rightarrow \sigma^{x}_{(x)} = \left(V^{\dagger}\right)^{-1}\sigma^{x}_{(z)}(V)^{-1}
    \end{align*}
    $\sigma^{x}_{(z)}$ is the matrix of $\sigma^{x}$ in the $\sigma^{z}$ representation. Noticed that $V = V^{\dagger} = V^{-1}$, so
    \begin{align*}
      \sigma^{x}_{(x)} = V\sigma^{x}_{(z)}V
    \end{align*}}}
  \item  \textbf{The three components of the spin angular momentum operator $\vec{S}$ for spin-$1/2$ are $S^{x}$, $S^{y}$, and $S^{z}$. If we use the $S^{z}$ representation, their matrix representations are given by $\begin{aligned}
    \vec{S} = \frac{\hbar}{2}\vec{\sigma}
  \end{aligned}$, where the three components of $\vec{\sigma}$ are the Pauli matrices $\sigma^{x}$, $\sigma^{y}$, and $\sigma^{z}$.}
  
  \textbf{Now consider using the $S^{x}$ representation. Please list the order of basis vectors you have chosen in the $S^{x}$ representation, and calculate the matrix representations of the three components of the operator $\vec{S}$ in this representation.}

  {\color{gray}{Within $S^{z}$ representation, we have
  \begin{align*}
    S^{x}_{(z)} = \frac{\hbar}{2}\sigma^{x}_{(z)} = \frac{\hbar}{2}\begin{pmatrix}
      0 & 1 \\
      1 & 0
    \end{pmatrix}
  \end{align*}
  From the previous question, we have found the eigenvalues and corresponding eigenvectors:
  \begin{align*}
    |+\rangle_{x} = \frac{1}{\sqrt{2}}\begin{pmatrix}
      1\\
      1
    \end{pmatrix},\quad |-\rangle_{x} = \frac{1}{\sqrt{2}}\begin{pmatrix}
      1\\
      -1
    \end{pmatrix}.
  \end{align*}
  The matrix $V$ that transforms the $S^{z}$ representation to the $S^{x}$ representation is 
  \begin{align*}
    V = \frac{1}{\sqrt{2}}\begin{pmatrix}
      1 & 1 \\
      1 & -1
    \end{pmatrix}
  \end{align*}
  In the $S^{z}$ representation, we have
  \begin{align*}
    S^{x}_{(z)} = \frac{\hbar}{2}\sigma^{x} = \frac{\hbar}{2}\begin{pmatrix}
      0 & 1 \\
      1 & 0
    \end{pmatrix},\quad
    S^{y}_{(z)} = \frac{\hbar}{2}\sigma^{y} = \frac{\hbar}{2}\begin{pmatrix}
      0 & -i \\
      i & 0
    \end{pmatrix},\quad
    S^{z}_{(z)} = \frac{\hbar}{2}\sigma^{z} = \frac{\hbar}{2}\begin{pmatrix}
      1 & 0 \\
      0 & -1
    \end{pmatrix}.
  \end{align*}
  So
  \begin{align*}
    S^{x}_{(x)} &= VS^{x}_{(z)}V = \frac{\hbar}{2}\begin{pmatrix}
      1&0\\
      0&-1
    \end{pmatrix},\\ 
    S^{y}_{(x)} &= VS^{y}_{(z)}V = \frac{1}{\sqrt{2}}\begin{pmatrix}
      1 & 1 \\
      1 & -1
    \end{pmatrix}\frac{\hbar}{2}\begin{pmatrix}
      0 & -i \\
      i & 0
    \end{pmatrix}\frac{1}{\sqrt{2}}\begin{pmatrix}
      1 & 1 \\
      1 & -1
    \end{pmatrix} = \frac{\hbar}{2}\begin{pmatrix}
      0 & -i \\
      i & 0
    \end{pmatrix},\\
    S^{z}_{(x)} &= VS^{z}_{(z)}V = \frac{1}{\sqrt{2}}\begin{pmatrix}
      1 & 1 \\
      1 & -1
    \end{pmatrix}\frac{\hbar}{2}\begin{pmatrix}
      1 & 0 \\
      0 & -1
    \end{pmatrix}\frac{1}{\sqrt{2}}\begin{pmatrix}
      1 & 1 \\
      1 & -1
    \end{pmatrix} = \frac{\hbar}{2}\begin{pmatrix}
      0 & 1\\
      1 & 0
    \end{pmatrix}.
  \end{align*}
  So the basis vectors in the $S^{x}$ representation are 
  \begin{align*}
    |+\rangle_{(x)}^{x} = \begin{pmatrix}
      1\\
      0
    \end{pmatrix},\quad |-\rangle_{(x)}^{x} = \begin{pmatrix}
      0\\
      1
    \end{pmatrix}.
  \end{align*}}}
\end{enumerate}
\end{document}