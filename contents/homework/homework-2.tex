\documentclass[../../main.tex]{subfiles}
\graphicspath{{\subfix{../images/}}} % 指定图片目录,后续可以直接使用图片文件名。
\begin{document}
\section{Homework 2}
\subsection{Angular momentum for 4-dimensional space}
\textbf{Consider a 4-dimensional space with coordinates $(x, y, z, w)$.}
\begin{enumerate}
  \item \textbf{Show that the operators $L_{i} = \epsilon_{ijk}x_{j}p_{k}$ and $K_{i} = wp_{i}-x_{i}p_{w}$ generate rotations in this space by showing that the transformations generated by these operators leave the four dimensional radius, defined by $R^{2} = x^{2} + y^{2} + z^{2} + w^{2}$, invariant.}
  %通过证明算子 $L_{i} = \epsilon_{ijk}x_{j}p_{k}$ 和算子 $K_{i} = wp_{i}-x_{i}p_{w}$ 产生的变换对于四维半径(定义为 $R^{2} = x^{2} + y^{2} + z^{2} + w^{2}$ 具有不变性。
{\color{white}{  \begin{enumerate}
    \item Since the operator $\begin{aligned}
      L_{i} = \sum_{jk}\epsilon_{ijk}x_{j}p_{k}
    \end{aligned}$ is defined in the usual 3-dimension subspace, so we still have
    \begin{align*}
      [L_{i},x_{j}] &= \left[\sum_{kl}\epsilon_{ikl}x_{k}p_{l},x_{j}\right] = \sum_{kl}\epsilon_{ikl}[x_{k}p_{l},x_{j}]\\
      &= \sum_{kl}\epsilon_{ikl}(x_{k}[p_{l},x_{j}] + \cancel{[x_{k},x_{j}]p_{l}}) = \sum_{kl}\epsilon_{ikl}x_{k}(-i\hbar\delta_{lj})\\
      &= \sum_{k}\epsilon_{ikj}x_{k}(-i\hbar) = \boxed{i\hbar\sum_{k}\epsilon_{ijk}x_{k}}.
    \end{align*}
    So we have
    \begin{align*}
      [L_{i},R^{2}] &= [L_{i},x^{2}+y^{2}+z^{2}+w^{2}] = [L_{i},x^{2}] + [L_{i},y^{2}] + [L_{i},z^{2}] + [L_{i},w^{2}],\\
      [L_{i},x_{j}^{2}] &= [L_{i},x_{j}x_{j}] = x_{j}[L_{i},x_{j}] + [L_{i},x_{j}]x_{j} = x_{j}\left[i\hbar\sum_{k}\epsilon_{ijk}x_{k}\right] + \left[i\hbar\sum_{k}\epsilon_{ijk}x_{k}\right]x_{j} \\
      &= 2i\hbar\sum_{k}\epsilon_{ijk}x_{j}x_{k}\\
      \left[L_{i},\sum_{j}^{3}x_{j}^{2}\right] &= \sum_{j}^{3}[L_{i},x_{j}^{2}] = 2i\hbar\sum_{jk}\epsilon_{ijk}x_{j}x_{k} = 0,\quad\text{since }j\leftrightarrow k\text{ symmetry}\\
      [L_{i},w^{2}] &= [L_{i},ww] = w[L_{i},w] + [L_{i},w]w= 0.
    \end{align*}
    So we have $[L_{i},R^{2}] = 0$, which means the operator $L_{i}$ leaves the 4-dimension radius invariant.
    \item $K_{i} = wp_{i} - x_{i}p_{w}$.
    
    Now we consider the commutator. Due to the definition of $K_{i}$, only the terms with $w$ will be affected. So we have:
    \begin{align*}
      [K_{i},R^{2}] &= [K_{i},x^{2} + y^{2} + z^{2} + w^{2}]= \sum_{j}^{3}[K_{i},x_{j}^{2}] + [K_{i},w^{2}]\\
      [K_{i},w^{2}] &= [K_{i},w]w + w[K_{i},w]\\
      [K_{i},w] &= [wp_{i} - x_{i}p_{w},w] = \left[w\left(-i\hbar\frac{\partial }{\partial x_{i}}\right)-x_{i}\left(-i\hbar\frac{\partial }{\partial w}\right),w\right]
    \end{align*}
    Assume a sample function $f(x,y,z,w)$, wo we have
    \begin{align*}
      &\left[w\left(-i\hbar\frac{\partial }{\partial x_{i}}\right)-x_{i}\left(-i\hbar\frac{\partial }{\partial w}\right),w\right]f = (-i\hbar)\left[w\frac{\partial }{\partial x_{i}} - x_{i}\frac{\partial }{\partial w},w\right]f\\
      &= (-i\hbar)\left\{\left(w\frac{\partial }{\partial x_{i}} - x_{i}\frac{\partial }{\partial w}\right)(wf) - w\left(w\frac{\partial f}{\partial x_{i}} - x_{i}\frac{\partial f}{\partial w}\right)\right\}\\
      &= (-i\hbar)(-x_{i})f\\
      &\Rightarrow \boxed{[K_{i},w] = i\hbar x_{i}}
    \end{align*}
    So we have
    \begin{align*}
      [K_{i},w^{2}] = [K_{i},w]w + w[K_{i},w] = i\hbar x_{i}w + w(i\hbar x_{i}) = 2i\hbar x_{i}w
    \end{align*}
    For the other term, we have
    \begin{align*}
      [K_{i},x_{j}] &= w[p_{i},x_{j}] =(-i\hbar)w\delta_{ij}\\
      [K_{i},x_{j}^{2}] &= [K_{i},x_{j}x_{j}] = x_{j}[K_{i},x_{j}] + [K_{i},x_{j}]x_{j} = -2i\hbar x_{j}w\delta_{ij}
    \end{align*}
    Thus we have
    \begin{align*}
      [K_{i},R^{2}] = [K_{i},x^{2} + y^{2} + z^{2} + w^{2}] = \sum_{j}^{3}\left[2i\hbar x_{j}w\delta_{ij}\right] - 2i\hbar x_{i}w = 2i\hbar x_{i}w - 2i\hbar x_{i}w = 0.\\\square
    \end{align*}
  \end{enumerate}}}

  \item \textbf{Compute the commutators $[L_{i}, K_{j}]$ and $[K_{i}, K_{j}]$.}
{\color{white}{  \begin{enumerate}
    \item $[L_{i},K_{j}]$
    \begin{align*}
      [L_{i},K_{j}] &= [L_{i},wp_{j}-x_{j}p_{w}] = [L_{i},wp_{j}] - [L_{i},x_{j}p_{w}] = w[L_{i},p_{j}] - [L_{i},x_{j}p_{w}]
    \end{align*}
    We have known that $[p_{k},p_{j}]=0$ and $[x_{l},p_{j}] = i\hbar\delta_{lj}$, so we have
    \begin{align*}
      [L_{i},p_{j}] &= \left[\sum_{lk}\epsilon_{ilk}x_{l}p_{k},p_{j}\right] = 
      \sum_{lk}\epsilon_{ilk} (\cancel{x_{l}[p_{k},p_{j}]} +[x_{l},p_{j}]p_{k}) 
      = \sum_{lk}\epsilon_{ilk}i\hbar\delta_{lj} p_{k} = i\hbar\sum_{k}\epsilon_{ijk} p_{k}\\
      &\Rightarrow \boxed{w[L_{i},p_{j}] = i\hbar \sum_{k}\epsilon_{ijk} wp_{k}}
    \end{align*}
    For the other term, we have
    \begin{align*}
      [L_{i},x_{j}p_{w}] &= x_{j}[L_{i},p_{w}] + [L_{i},x_{j}]p_{w}\\
      [L_{i},x_{j}] &= \left[\sum_{kl}\epsilon_{ikl}x_{k}p_{l},x_{j}\right] = \sum_{kl}\epsilon_{ikl}[x_{k}p_{l},x_{j}]\\
      &= \sum_{kl}\epsilon_{ikl}(x_{k}[p_{l},x_{j}] + \cancel{[x_{k},x_{j}]p_{l}}) = \sum_{kl}\epsilon_{ikl}x_{k}(-i\hbar\delta_{lj})\\
      &= \sum_{k}\epsilon_{ikj}x_{k}(-i\hbar) = i\hbar\sum_{k}\epsilon_{ijk}x_{k},\\
      [L_{i},p_{w}] &= \sum_{jk}\epsilon_{ijk}[x_{j}p_{k},p_{w}] = \sum_{jk}\epsilon_{ijk}(x_{j}[p_{k},p_{w}] + [x_{j},p_{w}]p_{k}) = \epsilon_{ijk}(x_{j}\cdot 0 + 0\cdot p_{k}) = 0\\
      &\Rightarrow [L_{i},x_{j}p_{w}] = x_{j}\cdot 0 + i\hbar\sum_{k}\epsilon_{ijk}x_{k}\cdot p_{w} = \boxed{i\hbar\sum_{k}\epsilon_{ijk}x_{k}p_{w}}
    \end{align*}
    Combining the terms we derived, we have 
    \begin{align*}
      [L_{i},K_{j}] = i\hbar \sum_{k}\epsilon_{ijk} wp_{k} - i\hbar\sum_{k}\epsilon_{ijk}x_{k}p_{w} = \boxed{i\hbar\sum_{k}\epsilon_{ijk}K_{k}}
    \end{align*}
    \item $[K_{i},K_{j}]$. 
    \begin{align*}
      [K_{i},K_{j}] &= [wp_{i}-x_{i}p_{w},wp_{j}-x_{j}p_{w}] = [wp_{i},wp_{j}] - [wp_{i},x_{j}p_{w}] - [x_{i}p_{w},wp_{j}] + [x_{i}p_{w},x_{j}p_{w}]\\
      [wp_{i},wp_{j}] &= w^{2}[p_{i},p_{j}] = 0;\\
      [wp_{i},x_{j}p_{w}] &= x_{j}(\cancel{w[p_{i},p_{w}]} + [w,p_{w}]p_{i}) + (w[p_{i},x_{j}] + \cancel{[w,x_{j}]p_{i}})p_{w} = x_{j} i\hbar p_{i} + w (-i\hbar)\delta_{ij}p_{w}\\
      & = i\hbar(x_{j}p_{i}-\delta_{ij}wp_{w})\\
      [x_{i}p_{w},wp_{j}] &= w(\cancel{x_{i}[p_{w},p_{j}]} + [x_{i},p_{j}]p_{w}) + (x_{i}[p_{w},w] + \cancel{[x_{i},w]p_{w}})p_{j} = w i\hbar \delta_{ij}p_{w} + x_{i}(-i\hbar)p_{j}\\
      &= i\hbar(wp_{w}\delta_{ij}-x_{i}p_{j})\\
      [x_{i}p_{w},x_{j}p_{w}] &= 0
    \end{align*}
    So combine the terms we derived, we have
    \begin{align*}
      [K_{i},K_{j}] = 0 - i\hbar(x_{j}p_{i}\cancel{-\delta_{ij}wp_{w}}) - i\hbar(\cancel{wp_{w}\delta_{ij}}-x_{i}p_{j}) + 0 = i\hbar (x_{i}p_{j}-x_{j}p_{i}) = \boxed{i\hbar\sum_{k}\epsilon_{ijk}L_{k}}
    \end{align*}
  \end{enumerate}}}
\end{enumerate}

\subsection{Harmonic oscillator}
\begin{enumerate}
  \item \textbf{Find the energy eigenvalues $E_{n}$ and the corresponding wave functions $\psi_{n}(x)$ for a one-dimensional quantum harmonic oscillator system.}
  
{\color{white}{  We have known that the Hamitonian of a quantum harmonic oscillator is given by
  \begin{align*}
    \hat{H} = -\frac{\hbar^{2}}{2m}\frac{\mathrm{d}^{2}}{\mathrm{d}x^{2}} + \frac{1}{2}m\omega^{2}x^{2}
  \end{align*}
  And the energy eigenvalues $E_{n}$ are given by
  \begin{align*}
    \boxed{E_{n} = \left(n+\frac{1}{2}\right)\hbar\omega,\quad n = 0,1,2,\cdots}
  \end{align*}
  The corresponding wave functions $\psi_{n}(x)$ are given by
  \begin{align*}
    \boxed{\psi_{n}(x) = \frac{1}{\sqrt{2^{n}n!}}\left(\frac{m\omega}{\pi\hbar}\right)^{\frac{1}{4}}e^{-\frac{m\omega x^{2}}{2\hbar}}H_{n}\left(\sqrt{\frac{m\omega}{\hbar}}x\right)}
  \end{align*}
  where $H_{n}(x)$ are the Hermite polynomials.}}
  
  \item \textbf{Calculate $\langle m|x|n\rangle$, $\langle m|p|n \rangle$, $\langle m|x^{2}|n \rangle$, and $\langle m|p^{2}|n \rangle$.}
  
{\color{white}{  We have known that the position operator $x$ and the momentum operator $p$ could be expressed by the creation $a^{\dagger}$ and annihilation $a$ operators:
  \begin{align*}
    \hat{x} &= \sqrt{\frac{\hbar}{2m\omega}}\left(a + a^{\dagger}\right), \quad \hat{p} = i\sqrt{\frac{\hbar m\omega}{2}}\left(a^{\dagger} - a\right)\\
    \hat{x}^{2} &= \frac{\hbar}{2m\omega}(a+a^{\dagger})^{2} = \frac{\hbar}{2m\omega}(a^{2} + a^{\dagger 2} + a^{\dagger}a + aa^{\dagger})\\
    \hat{p}^{2} &= -\frac{\hbar m\omega}{2}(a^{\dagger} - a)^{2} = -\frac{\hbar m\omega}{2}(a^{\dagger 2} - a^{\dagger}a - aa^{\dagger} + a^{2})
  \end{align*}
  which is governed by
  \begin{align*}
    a|n\rangle = \sqrt{n}|n-1\rangle,\quad a^{\dagger}|n\rangle = \sqrt{n+1}|n+1\rangle
  \end{align*}
  Apply the calculating formula to the matrix elements, and we have
  \begin{align*}
    \langle m|\hat{x}|n\rangle &= \sqrt{\frac{\hbar}{2m\omega}}\left(\langle m|a|n\rangle + \langle m|a^{\dagger}|n\rangle\right) = \sqrt{\frac{\hbar}{2m\omega}}\left(\langle m|\sqrt{n}|n-1\rangle + \langle m|\sqrt{n+1}|n+1\rangle\right)\\
    &= \boxed{\sqrt{\frac{\hbar}{2m\omega}}(\sqrt{n}\delta_{m,n-1} + \sqrt{n+1}\delta_{m,n+1})}\\
    \langle m|\hat{p}|n\rangle &= i\sqrt{\frac{\hbar m\omega}{2}}(\langle m|a^{\dagger}|n\rangle - \langle m|a|n\rangle) = i\sqrt{\frac{\hbar m\omega}{2}}(\langle m|\sqrt{n+1}|n+1\rangle - \langle m|\sqrt{n}|n-1\rangle)\\
    &= \boxed{i\sqrt{\frac{\hbar m\omega}{2}}(\sqrt{n+1}\delta_{m,n+1} - \sqrt{n}\delta_{m,n-1})}\\
    \langle m|\hat{x}^{2}|n\rangle &= \frac{\hbar}{2m\omega}(\langle m|a^{2}|n\rangle + \langle m|a^{\dagger 2}|n\rangle + \langle m|a^{\dagger}a|n\rangle + \langle m|aa^{\dagger}|n\rangle)\\
    &= \frac{\hbar}{2m\omega}(\langle m|\sqrt{n(n-1)}|n-2\rangle + \langle m|\sqrt{(n+1)(n+2)}|n+2\rangle + \langle m|n|n\rangle + \langle m|n+1|n\rangle)\\
    &= \boxed{\frac{\hbar}{2m\omega}(\sqrt{n(n-1)}\delta_{m,n-2} + \sqrt{(n+1)(n+2)}\delta_{m,n+2} + n\delta_{m,n} + (2n+1)\delta_{m,n})}\\
    \langle m|\hat{p}^{2}|n\rangle &= -\frac{\hbar m\omega}{2}\left(\langle m|a^{\dagger 2}|n\rangle - \langle m|2a^{\dagger}a|n\rangle + \langle m|a^{2}|n\rangle - \langle m|1|n\rangle\right)\\
    &= \boxed{-\frac{\hbar m\omega}{2}(\sqrt{(n+1)(n+2)}\delta_{m,n+2} - (2n+1)2n\delta_{m,n} + \sqrt{n(n-1)}\delta_{m,n-2})}
  \end{align*}}}
  
  \item \textbf{Assume the quantum harmonic oscillator is in a thermal bath at temperature $T$; find the partition function $Z$ and the average energy $\langle E\rangle$ of the system.}
  
{\color{white}{  Note $\begin{aligned}
    \frac{1}{k_{B}T}
  \end{aligned}$ as $\beta$ for simplicity. Since the energy eigenvalues are given by $\begin{aligned}
    E_{n} = \left(n+\frac{1}{2}\right)\hbar\omega
  \end{aligned}$, the partition function $Z$ is given by
  \begin{align*}
    Z = \sum_{n=0}^{\infty}e^{-\beta E_{n}} = \sum_{n=0}^{\infty}e^{-\beta\left(n+\frac{1}{2}\right)\hbar\omega} = e^{-\frac{1}{2}\beta\hbar\omega}\sum_{n=0}^{\infty}e^{-\beta\hbar\omega n}
  \end{align*}
  For the series $\begin{aligned}
    \sum_{n=0}^{\infty}x^{n}
  \end{aligned}$, we have the limit value $\begin{aligned}
    \frac{1}{1-x}
  \end{aligned}$ when $|x|<1$. So we have
  \begin{align*}
    Z = e^{-\frac{1}{2}\beta\hbar\omega}\frac{1}{1-e^{-\beta\hbar\omega}} = \boxed{\frac{e^{-\frac{1}{2}\beta\hbar\omega}}{1-e^{-\beta\hbar\omega}}}
  \end{align*}
  The average energy $\langle E\rangle$ is given by
  \begin{align*}
    \langle E\rangle &= -\frac{\partial \ln Z}{\partial\beta} = -\frac{\partial}{\partial \beta}\left(-\frac{1}{2}\beta\hbar\omega - \ln(1 - e^{-\beta\hbar\omega})\right)\\
    &= -\left(-\frac{1}{2}\hbar\omega - \frac{1}{1-e^{-\beta\hbar\omega}}(-e^{-\beta\hbar\omega})(-\hbar\omega)\right)\\
    &= \boxed{\frac{1}{2}\hbar\omega + \frac{\hbar\omega }{e^{\beta\hbar\omega} - 1}}
  \end{align*}}}
  
  \item \textbf{Prove that the inner product of coherent states is given by:
  \begin{align*}
    \langle \alpha|\beta \rangle = e^{-\frac{1}{2}(|\alpha|^{2}+|\beta|^{2}) + \alpha^{*}\beta}
  \end{align*}}

{\color{white}{  The coherent states are given by
  \begin{align*}
    |\alpha\rangle &= e^{-\frac{1}{2}|\alpha|^{2}}\sum_{n=0}^{\infty}\frac{\alpha^{n}}{\sqrt{n!}}|n\rangle\\
    |\beta\rangle &= e^{-\frac{1}{2}|\beta|^{2}}\sum_{m=0}^{\infty}\frac{\beta^{m}}{\sqrt{m!}}|m\rangle
  \end{align*}
  So the inner product could be derived as
  \begin{align*}
    \langle \alpha|\beta\rangle &= \left(e^{-\frac{1}{2}|\alpha|^{2}}\sum_{n=0}^{\infty}\frac{\alpha^{*n}}{\sqrt{n!}}\langle n|\right)\left(e^{-\frac{1}{2}|\beta|^{2}}\sum_{m=0}^{\infty}\frac{\beta^{m}}{\sqrt{m!}}|m\rangle\right)\\
    &=e^{-\frac{1}{2}|\alpha|^{2}}e^{-\frac{1}{2}|\beta|^{2}}\sum_{n = 0}^{\infty}\sum_{m=0}^{\infty}\frac{(\alpha^{*})^{n}\beta^{m}}{\sqrt{n!m!}}\langle n|m\rangle
  \end{align*}
  where $\langle n|m\rangle = \delta_{n,m}$ due to the orthogonality of the energy eigenstates. So we have
  \begin{align*}
    \langle \alpha|\beta\rangle = e^{-\frac{|\alpha|^{2}}{2}}e^{-\frac{|\beta|^{2}}{2}}\sum_{n=0}^{\infty}\frac{\alpha^{*n}\beta^{n}}{n!} = e^{-\frac{|\alpha|^{2}+|\beta|^{2}}{2}}\sum_{n=0}^{\infty}\frac{(\alpha^{*}\beta)^{n}}{n!} = e^{-\frac{|\alpha|^{2}+|\beta|^{2}}{2}}e^{\alpha^{*}\beta}.\quad\square
  \end{align*}}}
\end{enumerate}

\end{document}